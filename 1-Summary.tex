\begin{center}
\noindent\textbf{\large The Normative Relations between\\ Fiction, Imagination and
Appreciation}
\end{center}


\noindent\textbf{\large 1. Summary}
\vspace{.1cm}

\noindent It is a commonplace that fictional texts --- notably in the shape of novels and other works of literature --- invite and move us to imagine certain things concerning the characters and events described by the text. Reading \emph{The Sound and the Fury}, say, prompts us to imagine various incidents in the history of the Compson family. It is equally undeniable that considerations about what fictional texts ask us to imagine, and how, are central to our aesthetic appreciation of them. We value Faulkner's novel in part because it allows us to imagine these incidents from the very complex, diverse and subjective points of view of the three Compson brothers.

What these two truisms about the relationship between fiction, imagination and appreciation have in common is that they describe \emph{normative} relations. Imagining is the appropriate basic response to fictional texts. We cannot understand fictional texts without becoming imaginatively involved with the world that they describe. If we fail to imagine in accordance with the text's prescriptions to imagine, we fail to properly engage with it. Similarly, appreciation is the appropriate response to the power of fictional texts to captivate our imagination in rich and rewarding ways. Much of the aesthetic worth of fictional texts resides in the fact that they make specific fictional worlds accessible to our imagination. Hence, the aesthetic evaluation of fictional texts requires us to take into account what they ask us to imagine, and how they do this.

The general aim of this interdisciplinary research project is to investigate the nature of these two normative relations, which may be characterized in the following general way:


\vspace{-.1cm}
\begin{itemize}[leftmargin=2cm]
\item[{(NR1)}] Fictional texts direct us to imagine certain things.
\vspace{-.2cm}
\item[{(NR2)}] That fictional texts direct us to imagine certain things bears on whether we should aesthetically (dis)value them. 
\end{itemize}
\vspace{-.1cm}


\noindent With respect to each fictional text, there are specific instances of (NR1) and (NR2), which tell us what to imagine when reading the text, and also how to aesthetically assess it in light of what it asks us to imagine. Although certain important aspects of (NR1) and (NR2) have already been discussed in considerable detail in the literature (e.g. how it is generally determined what is fictional relative to a given text), many others have not (e.g. what the value of fiction-based imagining is, or what literary scholars actually think about these normative issues). Thus there has so far been no systematic investigation of the normative role of imagining and its normative connections to fiction and appreciation --- something that this research projects aims to remedy. 

The neglected issues that we intend to address can be divided into factual, normative and foundational questions (not unlike the division into applied ethics, normative ethics and metaethics). In line with this division, our research project consists of three subprojects. \emph{Subproject 1} --- to be led by Tom Kindt (Literary Studies/Fribourg) --- inquires into which specific instances of (NR1) and (NR2) are \emph{de facto} accepted by literary scholars in their interpretative practice (with a focus on E.T.A. Hoffmann's `Der Sandmann'). \emph{Subproject 2} --- to be directed by Tilmann K\"oppe (Literary Theory/G\"ottingen) --- investigates which particular instances of (NR1) and (NR2) \emph{should} (or \emph{should not}) guide our interpretation of fictional texts. And \emph{Subproject 3} -- to be run by Fabian Dorsch (Philosophy/Fribourg) --- intends to answer why these instances do possess normative authority over our imaginative and appreciative engagement with fictional texts. 

Together, the three subprojects aim at providing a comprehensive account of the normative relations between fiction, imagination and appreciation accross disciplines. So far, research in literary studies and philosophy displays a picture of complementary strengths and weaknesses. While philosophical aesthetics has contributed to our theoretical understanding of fiction, imagination and appreciation, it has all too often neglected actual interpretative practice. The interpretations conducted in literary studies, in turn, host a wealth of insights concerning our engagement with the arts. The insights, however, need to be made explicit and clarified for the purpose of theory building. While this may amount to a truism that has often led to a call for joint efforts of the disciplines, such endeavors are but seldomly undertaken. The interdisciplinary approach of the proposed research promises to combine the strengths of both philosophy and literary studies, while mend their blind spots. 

\pagebreak