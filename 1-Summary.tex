\begin{center}
\noindent\textbf{\large The Normative Relations between\\ Fiction, Imagination and
Appreciation}
\end{center}


\noindent\textbf{\large 1. Summary}
\vspace{.1cm}

\noindent \textbf{\emph{The Main Topic.}} It is a commonplace that fictional texts --- notably in the shape of novels and other works of literature --- invite and move us to imagine certain things concerning the characters and events described by the text. Reading \emph{The Sound and the Fury}, say, prompts us to imagine various incidents in the history of the Compson family. It is equally undeniable that considerations about what fictional texts ask us to imagine, and how, are central to our aesthetic appreciation of them. We value Faulkner's novel in part because it allows us to imagine these incidents from the very complex, diverse and subjective points of view of the three Compson brothers.

What these two truisms about the relationship between fiction, imagination and appreciation have in common is that they describe two \emph{normative} relations. Imagining is the appropriate basic response to fictional texts. We cannot understand fictional texts without becoming imaginatively involved with the world that they describe. If we fail to imagine in accordance with the text's prescriptions to imagine, we fail to properly engage with it. Similarly, appreciation is the appropriate response to the power of fictional texts to captivate our imagination in rich and rewarding ways. Much of the aesthetic worth of fictional texts resides in the fact that they make specific fictional worlds in certain ways accessible to our imagination. Hence, the aesthetic evaluation of fictional texts requires us to take into account what they ask us to imagine, and how they want us to imagine it.

The general aim of the research project is to investigate the nature of these two normative relations between fiction, imagination and appreciation, which may be characterized in the following general way:


\vspace{-.1cm}
\begin{itemize}[leftmargin=2cm]
\item[{(NR1)}] Fictional texts direct us to imagine certain things.
\vspace{-.2cm}
\item[{(NR2)}] That fictional texts direct us to imagine certain things bears on whether we should aesthetically (dis)value them. 
\end{itemize}
\vspace{-.1cm}


\noindent With respect to each fictional text, there are specific instances of (NR1) and (NR2), which tell us what to imagine when reading the text, and also how to aesthetically assess it in light of what it asks us to imagine.

Although certain important aspects of (NR1) and (NR2) have already been discussed in detail in the literature (e.g. how it is generally determined what is fictional relative to a given text), many others have not (e.g. what the value of fiction-based imagining is, or what literary scholars actually think about these normative issues). A bit surprisingly, there has so far been no comprehensive, systematic investigation of the normative role of imagining and its normative connections to fiction and appreciation --- something that this research projects aims to remedy. 

The neglected issues that we intend to address can be divided into factual, normative and foundational questions (not unlike the division into applied ethics, normative ethics and metaethics). In line with this division, our research project consists of three subprojects.

\vspace{.2cm}
\noindent\textbf{\emph{Subproject 1: Factual Issues (Literary Studies / Fribourg).}} The first subproject explores the interrelation between imagination and appreciation in the practice of literary studies. By examining argumentative structures and evaluative components in a comprehensive sample of `expert interpretations' of a literary work, the subproject aims to gain empirical evidence for answering the following general question concerning the normative dimension of `doing interpretation' in literary criticism:

\vspace{-.1cm}
\begin{itemize}[leftmargin=2cm]
\item[(Q1)] Which specific instances of (NR1) and (NR2) are \emph{de facto} explicitly or implicitly accepted by literary scholars as governing our imaginative and appreciative engagement with fictional texts?
\end{itemize}
\vspace{-.1cm}

\noindent In particular, with respect to (NR1), the subproject studies the principles and rules that, in scholarly interpretations, form the actual basis for the imaginative construal of fictive worlds. And, concerning (NR2), it asks how the imagination of fictive worlds is in fact related to the appreciation of fictional works in the context of literary studies.

In order to answer these questions, the project will proceed in two steps: (1) Based on an evaluation of established attempts to illuminate the argumentative composition and evaluative alignment of critical discourse and to analyze the interplay of different operations involved in it, the first part of the subproject elucidates the basic categories and guiding procedures of the meta-critical study. (2) The second part develops an empirical account of the normative practice of interpretation in literary criticism, taking the extensive controversy elicited by E.T.A. Hoffmanns 1816-novella `Der Sandmann' as a test case. By analyzing a sample of about 60 contributions to this scholarly debate, the project aims, firstly, to uncover the norms that underlie the ascription of fictive facts and, secondly, to determine how the generation of fictive worlds is connected to the attribution of aesthetic value in the context in question.



\vspace{.2cm}
\noindent\textbf{\emph{Subproject 2: Normative Issues (Literary Theory / G\"ottingen).}}

\vspace{-.1cm}
\begin{itemize}[leftmargin=2cm]
\item[(Q2)] Which specific instances of (NR1) and (NR2) \emph{should} govern our imaginative and appreciative engagement with fictional texts?
\end{itemize}
\vspace{-.1cm}


\vspace{.2cm}
\noindent\textbf{\emph{Subproject 3: Foundational Issues (Philosophy / Fribourg).}} The third subproject aims to investigate why the normative relations between fiction, imagination and appreciation hold and what makes them possible. That is, it asks what is the source of their normative authority. Concerning (NR2), there are already a well-developed views on the normativity of aesthetic appreciation and thus also on how fiction-related prescriptions to imagine can constitute aesthetic qualities that are normatively relevant for appreciation (see *). For this reason, the subproject focuses instead on the much more neglected normativity of imagining and, especially, on (NR1): 

\vspace{-.1cm}
\begin{itemize}[leftmargin=2cm]
\item[(Q3)] What is the source of the normative authority of the specific instances of (NR1) over our imaginative engagement with fictional texts?
\end{itemize}
\vspace{-.1cm}

\noindent We have a good grasp of why assertive texts (e.g. reports) possess the normative power to direct our beliefs: because assertion and belief aim at the same value, namely truth (*). So, the question arises whether something similar holds of fiction and imagination; or whether the normative relation between the two has to be accounted for in different terms. So far, this issue has been primarily addressed in negative terms, in the shape of the observation that neither fiction, nor imagining are aimed at truth (*). Our goal is instead to provide a positive answer to (Q3). ***
% Our main hypothesis is that fiction and imagining do have a shared constitutive end, namely to exert direct control over what and how things are presented as being, free from any evidential constraints.



\vspace{.2cm}
\noindent\textbf{\emph{Institutional Setting.}} G\"ottingen is an ideal location for Subproject 2 due to the longstanding close cooperation between philosophy and literary studies established at the Courant Research Center `Textstrukturen' where Tilmann K\"oppe is head of the research group `Analytic literary theory' which currently hosts one postdoc and five doctoral students. Also, the German Department hosts the `Arbeitsstelle f\"ur Theorie der Literatur' (ATL), of which Tom Kindt and Tilmann K\"oppe are members. The ATL features a strong focus on analytic literary theory (cf. K\"oppe 2008a; K\"oppe/ Winko 2010) which makes it a natural cooperation partner for the planned project.