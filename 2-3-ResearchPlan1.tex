\noindent\textbf{\large 2.3. Detailed Research Plan}




\vspace{.2cm}
\noindent\textbf{\emph{Subproject 1: Factual Issues (Kindt/Literary Studies/Fribourg/PhD)}}
\vspace{.1cm}

\noindent The first subproject intends to answer (Q1) by concentrating on the particular case of the interpretative practice concerning E.T.A. Hoffmann's `Der Sandmann'. The smaller first part of the subproject addresses the task of shaping an analytic guideline for the intended empirical exploration of interpretation practices in literary criticism. The larger second part applies this guideline in a detailed meta-critical study of about 60 contributions to the interpretive controversy about `Der Sandmann'. Initially, the study aims to identify principles and rules that are at the bottom of the generation of fictional facts in literary studies. After that, it will examine the relations that are established in scholarly interpretations between the imagination of fictive worlds and the appreciation of the related fictional works.

%\vspace{.2cm}
%\noindent\textbf{An Analytic Guideline for an Exploration of Interpretive Practices.} 

To inform  the meta-critical analysis of the `Sandmann'-debate, the \textbf{first part} of the subproject will outline an `analytic manual' for the illumination of argumentative and evaluative structures in `expert interpretations'. With regard to \emph{argumentation analysis}, the study will follow Winko's proposal of adapting the Toulminean model of argumentation for examining procedures of justification and generating plausibility in literary critics' approaches to texts (Winko 2002, 2015a, 2015b). It is our contention that Toulmin's core distinction between `grounds' (i.e. reasons), `warrants' (i.e. inferential patterns or forms of reasoning), and `backings' (i.e. background justifications for these `warrants') for claims or conclusions (Toulmin 1958) facilitates a precise description of the complex and rarely elaborated argumentative dimension of scholarly interpretations and helps keeping in mind that explicit reasons and implicit regularities in argumentative practices do not necessarily refer to rules and principles governing the practices in question (von Savigny 1976). The \emph{evaluation analysis} of the corpus-texts will, in line with von Heydebrand and Winko, build on the speech act-account of interpretive practices in literary studies (Zabka 2005) and conceive of evaluation as a specific illocutionary act characterized by the attributive usage of `value-terms' (von Heydebrand/Winko 1996). Such a conception of evaluation draws a generally adequate picture of the ascription of aesthetic values in scholarly interpretations and has recently been the point of reference for a comprehensive catalogue of questions and procedures tailor-made to analyze evaluative practices in critical discourse (Worthmann 2013).

% \vspace{.2cm}
% \noindent\textbf{A Case Study on the Interpretive Debate about `Der Sandmann'.} 

Guided by the scheme of categories that have been clarified in the first part of the subproject, its \textbf{second and main part} is devoted to a meta-critical appraisal of the long-standing and still ongoing interpretive debate on E.T.A. Hoffmann's novella `Der Sandmann'. The corpus of the study will consist of about 60 contributions to the controversy that, by now, comprises approximately 100 elaborated scholarly interpretations. All elements of the sample to be analyzed have been published in the course of the last hundred years, that is, after the formation of literary studies in its present disciplinary setup. 

By examining the argumentative structures in the corpus-texts related to the issues of both `world construction' and `work evaluation', the subprobject intends to exemplarily elucidate central aspects of the normative practice of interpretation in literary criticism. More specifically, the second part of the subproject aims at formulating empirically underpinned answers to the following two, more specific questions:

\vspace{-.1cm}
\begin{itemize}[leftmargin=2cm]
\item[(Q1.1)] What are the principles and rules that, in the scholarly interpretations of `Der Sandmann', form the \emph{actual} basis for the imaginative construal of the fictional world of that work?
\end{itemize}
\vspace{-.1cm}

\vspace{-.1cm}
\begin{itemize}[leftmargin=2cm]
\item[(Q1.2)] How is the imaginative construal of this fictive world \emph{de facto} related to the appreciation of `Der Sandmann' in the scholarly practice?
\end{itemize}
\vspace{-.1cm}

\noindent The controversy over Hoffmann's `Sandmann' is ideally suited as a test case for the projected empirical study for at least three reasons. 

First, the debate has already been the object of several survey articles (e.g. Kremer 2010) and some meta-critical appraisals (Tepe/Rauter/Semlow 2009; Detel 2014). Since these publications convincingly identify basic camps and lines of argument in the controversy but give no noteworthy attention to the key questions of our intended study, they can serve as an advanced starting point for the realization of Subproject 1. 

Second, the debate about Hoffmann's text addresses the problem of determining the outlines of the fictive world. At the heart of the controversy, as the available surveys unanimously claim (Kindt 2016), is the question of whether the novella tells the story of a mental derangement due to psychic processes, or whether it alternatively decribes a fatal life crisis brought about by demonic powers, or whether it instead supplies insufficient evidence for determining what finally causes the death of the narrative's protagonist. In short, the interpretative debate about the text is essentially a dispute about the `elucidation' (Beardsley 1970) of the fictional world related to the text and, therefore, constitues a highly instructive test case for the subproject. 

Third, meta-critical publications on the `Sandmann'-controversy and random samples of the likely corpus of the study indicate that many contributions to the debate -- explicitly or implicitly -- intermingle the task of determining the fictional world of `Der Sandmann' with the task of aesthetically evaluating (aspects of) the fictional work. Thus, the debate provides a rich collection of examples of how `world construction' and `work evaluation' are related in scholary interpretations.

As discussed by the Subprojects 2 and 3, the orthodox view in fiction theory is that what is fictional relative to a given text consists in what is to be imagined in response to that text; and that imagination is a rule-governed practice which can be performed in an accurate or inaccurate way. Using the `Sandmann'-controversy as an example, the main part of Subproject 1 attempts to explore whether, and how, this picture of imagining fictional worlds is reflected in scholarly interpretations of fictional works.

However, since the contributions to the debate do not use the terminology of fiction and imagination theory, since surface and deep structures in literary criticism may diverge, and since meta-critical approaches to the field have repeatedly challenged the view that the activity of scholarly interpretation is governed by principles and rules (e.g. G\"ottner 1973; Schmidt 1979), the subproject will adopt a precautionary measure, before commencing with the thorough analysis. More specifically, in order to establish comparability between the various contributions to the debate and to assess their specific relevance for the subproject, the study will start with a detailed paraphrase of selected passages of the corpus-texts. Based on this paraphrase, the appraisal of the critical debate on Hoffmann's `Der Sandmann' can then proceed with a meta-critical analysis that successively addresses a catalogue of questions with regard to each interpretation in the corpus.

First of all, there are questions in relation to (Q1.1) and the imaginative construal of the fictional world of `Der Sandmann', which are concerned with \emph{ontological claims:}

\vspace{-.1cm}
\begin{itemize}[leftmargin=2cm]
\item What does the interpretation --- or INT, for short --- claim regarding the fictional world’s fundamental laws and structures?
\vspace{-.1cm}
\item How does INT determine the connection between the basic events of the plot?
\vspace{-.1cm}
\item How does INT construe the relation between, and the ontological status of, the characters Coppelius, Coppola and Sandmann? 
\end{itemize}
\vspace{-.3cm}

\noindent Then, there are also questions pertaining to (Q1.2) and the imagination-based appraisal of Hoffmann's novella, which are about \emph{aesthetic assessments:}

\vspace{-.1cm}
\begin{itemize}[leftmargin=2cm]
\item Does INT contain explicit or implicit aesthetic evaluations of (aspects of) the fictional work? 
\vspace{-.1cm}
\item If so, which elements of the fictional work does INT assess aesthetically, and in which specific way?
\vspace{-.1cm}
\item Are INT's aesthetic evaluations of the fictional work related to its claims about the fictional world?
\vspace{-.1cm}
\item Does INT assign to certain aspects of `world-construction' an argumentative role in the process of `work-apprecation'; or vice versa?
\vspace{-.1cm}
\item If so, which specific kind of argumentative role is assumed?
\end{itemize}
\vspace{-.3cm}

\noindent Finally, there are questions about the way in which the ontological claims about, and the aesthetic evaluations of, the text are \emph{argumentatively justified:}

\vspace{-.1cm}
\begin{itemize}[leftmargin=2cm]
\item Does INT provide justification for its answers to these questions? 
\vspace{-.1cm}
\item If so, what are the `grounds' (in Toulmin's sense) that INT offers for its claims regarding the fictional world?
\vspace{-.1cm}
\item Does INT refer additionally to `warrants' or `backings' (again in Toulmin's sense) to support its defense of its claims?
\vspace{-.1cm}
\item Does INT contain evidence that indicates its implicit acceptance and application of specific principles and rules for the construal of fictional worlds?
\end{itemize}
\vspace{-.3cm}

\noindent In addition to a systematically refined survey of the `Sandmann'-controversy, the analysis of the corpus-texts guided by this catalogue of analytic perspectives will undoubtedly render sufficient empirical evidence for developing a sound answer to the subproject's leading questions (Q1.1), (Q1.2) and, ultimately, (Q1) --- an answer that illuminates the normative dimension of the practice of interpretation.

