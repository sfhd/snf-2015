\noindent\textbf{\large 2.3. Detailed Research Plan}

\noindent While Subproject 1 aims to uncover normative principles that (\emph{de facto}) guide content ascribing interpretations as conducted by literary scholars, Subproject 2 examines arguments in support of such principles and Subproject 3 investigates the origin of the normative authority of these principles.


\vspace{.2cm}
\noindent\textbf{\emph{Subproject 1: Factual Issues (Literary Studies / Fribourg)}}
\vspace{.1cm}

This subproject addresses the following general question:

\vspace{-.1cm}
\begin{itemize}[leftmargin=2cm]
\item[(Q1)] Which specific instances of (NR1) and (NR2) are \emph{de facto} explicitly or implicitly accepted by literary scholars as governing our imaginative and appreciative engagement with fictional texts?
\end{itemize}
\vspace{-.1cm}


\noindent It intends to answer (Q1) by concentrating exclusively on the particular case of the interpretative practice concerning E.T.A. Hoffmann's `Der Sandmann'. The subproject divides into two parts. A brief first part addresses the task of shaping an analytic guideline for the intended empirical exploration of interpretation practices in literary criticism. The following main part of the project, divided into two subparts, applies the guideline in a detailed meta-critical study of about 60 contributions to the interpretive controversy about `Der Sandmann'. Initially, the study aims to identify principles and rules that are at the bottom of the generation of fictive facts in literary studies. After that, it will examine the relations that are established in scholarly interpretations between the imagination of fictive worlds and the appreciation of the related fictional works.

%\vspace{.2cm}
%\noindent\textbf{An Analytic Guideline for an Exploration of Interpretive Practices.} 

To inform  the meta-critical analysis of the `Sandmann'-debate, the \textbf{first part} of the subproject will outline an `analytic manual' for the illumination of argumentative and evaluative structures in `expert interpretations'. With regard to \emph{argumentation analysis}, the study will follow Winko's proposal of adapting the Toulminean model of argumentation for examining procedures of justification and generating plausibility in literary critics' approaches to texts (Winko 2002, 2015a, 2015b). Toulmin's core distinction between `grounds', `warrants', and `backings' for `claims' (Toulmin 1958), it will be argued, lays the ground for a precise description of the complex and rarely elaborated argumentative dimension of scholarly interpretations and helps keeping in mind that explicit reasons and implicit regularities in argumentative practices do not necessarily refer to rules and principles governing the practices in question (von Savigny 1976). The \emph{evaluation analysis} of the corpus-texts will, in line with von Heydebrand and Winko, build on the speech act-account of interpretive practices in literary studies (Zabka 2005) and conceive of evaluation as a specific illocutionary act characterized by the attributive usage of `value-terms' (von Heydebrand/Winko 1996). Such a conception of evaluation draws a generally adequate picture of the ascription of aesthetic values in scholarly interpretations and has recently been the point of reference for a comprehensive catalogue of questions and procedures tailor-made to analyze evaluative practices in critical discourse (Worthmann 2013).

% \vspace{.2cm}
% \noindent\textbf{A Case Study on the Interpretive Debate about `Der Sandmann'.} 

Guided by the scheme of categories that have been clarified in the first part of the subproject, its \textbf{second and main part} is devoted to a meta-critical appraisal of the long-standing and still ongoing interpretive debate on E.T.A. Hoffmann's novella `Der Sandmann'. The corpus of the study will consist of about 60 contributions to the controversy that, by now, comprises approximately 100 elaborated scholarly interpretations. All elements of the sample to be analyzed have been published in the course of the last hundred years, that is, after the formation of literary studies in its present disciplinary setup. 

By examining the argumentative structures in the corpus-texts related to the issues of both `world construction' and `work evaluation', the subprobject intends to exemplarily elucidate central aspects of the normative practice of interpretation in literary criticism. More specifically, each of the two subparts aims at the formulation of an empirically underpinned answer to one of the following specific questions:

\vspace{-.1cm}
\begin{itemize}[leftmargin=2cm]
\item[(Q 1.1)] What are the principles and rules that, in the scholarly interpretations of `Der Sandmann', form the \emph{actual} basis for the imaginative construal of the fictional world of that work?
\end{itemize}
\vspace{-.1cm}

\vspace{-.1cm}
\begin{itemize}[leftmargin=2cm]
\item[(Q 1.2)] How is the imaginative construal of this fictive world \emph{de facto} related to the appreciation of `Der Sandmann' in the scholarly practice?
\end{itemize}
\vspace{-.1cm}

\noindent The controversy over Hoffmann's `Sandmann' is ideally suited as a test case for the projected empirical study for at least three reasons. 

First, the debate has already been the object of several survey articles (e.g. Kremer 2010) and some meta-critical appraisals (Tepe/Rauter/Semlow 2009; Detel 2014). Since these publications convincingly identify basic camps and lines of argument in the controversy but give no noteworthy attention to the key questions of our intended study, they can serve as an advanced starting point for the realization of Subproject 1. 

Second, the debate about Hoffmann's text differs from comparable controversies in one distinctive respect. While many scholarly discussions about fictional works do not address the problem of determining the outlines of the fictive world, let alone deal with it at length, the `Sandmann'-debate is centered around that very problem. At the heart of the controversy, as the available surveys unanimously claim (Kindt, in press), is the question of whether the novella tells the story of a mental derangement due simply to psychic processes, or whether it alternatively decribes a fatal life crisis brought about by demonic powers, or whether it instead supplies insufficient evidence for determining what finally causes the death of the narrative's protagonist. In short, the interpretative debate about the text is essentially a dispute about the `elucidation' (Beardsley 1970) of the fictional world related to the text and, therefore, constitues a highly instructive test case for the subproject. 

Third, meta-critical publications on the `Sandmann'-controversy and random samples of the likely corpus of the study indicate that many contributions to the debate -- explicitly or implicitly -- *intermingle/combine the task of determining the fictional world of `Der Sandmann' with the task of aesthetically evaluating (aspects of) the fictional work. Thus, the debate provides a rich collection of examples of how `world construction' and `work evaluation' are related in scholary interpretations.