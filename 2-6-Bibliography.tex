\vspace{.6cm}
\noindent\textbf{\large Bibliography}
\vspace{.2cm}

\begin{hangparas}{.5cm}{1}

Albrecht, A. et al. (eds.) 2015: \emph{Theorien, Methoden und Praktiken des Interpretierens}, Berlin.

Bareis, J. 2008: \emph{Fiktionales Erz\"ahlen}, G\"oteborg.

--- 2009: `Was ist wahr in der Fiktion?', \emph{Scientia Poetica} 13, 230-254. 
Beardsley, M. C. 1970: \emph{The Possibility of Criticism}, Detroit.

Beetz, M. \& Meggle, G. 1976: \emph{Interpretationstheorie und Interpretationspraxis}, Kronberg.

Bender, J. 1995: `General but Defeasible Reasons in Aesthetic Evaluation', \emph{Journal of Aesthetics and Art Criticism} 53, 379-392.

Budd, M. 1995: \emph{Values of Art. Pictures, Poetry and Music}, London. 

Burge, T. 2010: \emph{Origins of Objectivity}, Oxford.

--- 2013a: `Content Preservation', in T. Burge, \emph{Cognition Through Understanding}, Oxford.

--- 2013b: `Postscript to `Content Preservation'', in T. Burge, \emph{Cognition Through Understanding}, Oxford.

Carroll, N. 1997: `Fiction, Non-Fiction, and the Film of Presumptive Assertion', in R. Allen \& M. Smith (eds.), \emph{Film Theory and Philosophy}, Oxford.

Choi, J. 2005: `Leaving It Up to the Imagination', \emph{The Journal of Aesthetics and Art Criticism} 63, 17--25. 

Currie, G. 1990: \emph{The Nature of Fiction}, Cambridge. 

--- 1991: `Interpreting Fictions', in R. Freadman and L. Reinhardt (eds.), \emph{On Literary Theory and Philosophy}, Basingstoke. 

--- 2003: `Interpretation in Art', in J. Levinson (ed.), \emph{The Oxford Handbook of Aesthetics}, Oxford.

Currie G. \& Ravenscroft I. 2002: \emph{Recreative Minds}, Oxford.

Davies, D. 2007: \emph{Aesthetics and Literature}, London.

Detel, W. 2014: \emph{Kognition, Parsen und rationale Erkl\"arung}, Frankfurt/M.

Diehl, N. 2009: `Imaginging De Re and the Symmetry Thesis of Narration', \emph{The Journal of Aesthetics and Art Criticism} 67, 15--24.

Dorsch, F. 2000: \emph{The Nature of Aesthetic Experiences}, London.
 
--- 2007: `Sentimentalism and the Intersubjectivity of Aesthetic Evaluations', \emph{dialectica} 61, 417-446. 
 
--- 2009: `Judging and the Scope of Mental Agency', in L. O'Brien \& M. Soteriou (eds.), \emph{Mental Actions}, Oxford. 
 
--- 2010: `Transparency and Imagining Seeing', \emph{Philosophical Explorations} 13, 173-200. 
 
--- 2011: `Emotional Imagining and Our Responses to Fiction', \emph{Enrahonar} 46, 153-176. 
 
--- 2012a: \emph{The Unity of Imagining}, Berlin.

--- 2012b: `Die Grenzen des \"asthetischen Empirismus', \emph{Zeitschrift f\"ur \"Asthetik und allgemeine Kunstwissenschaft} 57, 269-281.
 
--- 2013: `Non-Inferentialism about Justification --- the Case of Aesthetic Judgements', \emph{The Philosophical Quarterly} 63, 660-682.

--- 2014: `The Limits of Aesthetic Empiricism', in G. Currie, M. Kieran, A. Meskin \& J. Robson (eds.), \emph{Aesthetics and the Sciences of the Mind}, Oxford. 

--- 2015a: `Knowledge by Visualisation', under review.

--- 2015b: `Pictorial Experience, Imagining De Re, and Imaginative Penetration', under review.

--- 2016a: `Focused Daydreaming and Mind-Wandering', \emph{The Review of Philosophy and Psychology}, in press.
 
--- 2016b: `Seeing-In as Aspect Perception', in G. Kemp \& G. Mras (eds.), \emph{Seeing Something in Something: Wollheim, Wittgenstein, and Pictorial Representation}, London, in press.

--- 2016c: `Hume on the Imagination', in A. Kind (ed.), \emph{Handbook of the Philosophy of Imagination}, London, in press.
 
--- 2016d: `The Phenomenal Presence of Reasons', in F. Macpherson, M. Nida-R\"umelin \& F. Dorsch (eds.), \emph{Phenomenal Presence}, Oxford, forthcoming. 
 
--- 2016e: \emph{Imagination}, London, forthcoming.

Eco, U. 1999: \emph{Die Grenzen der Interpretation}, M\"unchen. 

Evans, G. 1982: \emph{The Varieties of Reference}, Oxford. 

Everett, A. 2013: \emph{The Nonexistent}, Oxford. 

Feagin, S. 1996: \emph{Reading with Feeling}, Ithaca.

F\o{}llesdal, D. \& Wall\o{}e L. \& Elster, J. 2008: `Die hypothetisch-deduktive Methode in der Literaturinterpretation', in T. Kindt \& T. K\"oppe (eds.), \emph{Moderne Interpretationstheorien}, G\"ottingen.

Fricke, H. 1977: \emph{Die Sprache der Literaturwissenschaft}, M\"unchen.

Gendler, T. 2000: `The Puzzle of Imaginative Resistance', \emph{Journal of Philosophy}, 97, 55-81.

Genette, G. 1980: \emph{Narrative Discourse}, New York. 

Gertken, J. \& K\"oppe, T. 2009: `Fiktionalit\"at', in F. Jannidis et al. (eds.), \emph{Grenzen der Literatur}, Berlin. 

Goldman, A. 1995: \emph{Aesthetic Value}, Boulder. 

--- 2006: `The Experiential Account of Aesthetic Value', \emph{The Journal of Aesthetics and Art Criticism} 64, 333-342.

G\"ottner, H. 1973: \emph{Logik der Interpretation}, M\"unchen.

Grewendorf, G. 1975: \emph{Argumentation und Interpretation}, Kronberg.

Habermas, T. 2006: `Who speaks? Who looks? Who feels? Point of view in autobiographical narratives', \emph{International Journal of Psychoanalysis} 87, 497-518. 

Heydebrand, R. von \& Winko, S. 1996: \emph{Einf\"uhrung in die Wertung von Literatur}, Paderborn.

Hieronymi, P. 2014: `Reflection and Responsibility', \emph{Philosophy and Public Affairs} 42, 3-41.

Kania, A. 2005: `Against the Ubiquity of Fictional Narrators', \emph{The Journal of Aesthetics and Art Criticism} 63, 47--54.

Kind, A. 2011: `The Puzzle of Imaginative Desire', \emph{Australasian Journal of Philosophy} 89, 421-439.

Kindt, T. 2007: `Denn sie wissen nicht, was sie tun. Stanley Fish vs. Wolfgang Iser', in R. Klausnitzer \& C. Spoerhase (eds.), \emph{Kontroversen in der Literaturtheorie / Literaturtheorie in der Kontroverse}, Berlin, 353-368.

--- 2008: \emph{Unzuverl\"assiges Erz\"ahlen und literarische Moderne. Eine Untersuchung der Romane von Ernst Wei{\ss}}, T\"ubingen. 

--- 2011: `\,`Das Unm\"ogliche, das dennoch geschieht'. Zum Begriff der literarischen Phantastik', in \emph{Thomas Mann-Jahrbuch} 24, 43-56.

--- 2015a: `Deskription und Interpretation. Handlungstheoretische und praxeologische Reflexionen', in M. Lessing-Sattarie, M. L\"ohden \& D. Wieser (eds.), \emph{Interpretationskulturen}, Berlin, 93-112.

--- 2015b: `Epoche machen! Zur Verteidigung eines umstrittenen Begriffs der Literaturgeschichte', in D. Fulda, S. Kerschbaumer \& S. Matuschek (eds.), \emph{Aufkl\"arung -- Romantik. Schnittpunkte zweier Modernekonstituenten}, M\"unchen, 11-22.

--- 2016: `\,`Erz\"ahlend immer mehr und mehr Farbe hineinzutragen?' E.T.A. Hoffmanns \emph{Der Sandmann} aus narratologischer Perspektive', in O. Jahraus (ed.): \emph{E. T. A. Hoffmnns „Der Sandmann` und die Literaturtheorie}, Stuttgart, in press.

Kindt, T. \& M\"uller H. 2003: `Narrative Theory and/or/as Theory of Interpretation', in T. Kindt \& H. M\"uller (eds.), \emph{What Is Narratology? Questions and Answers Regarding the Status of a Theory}, Berlin.

--- 2005: `Nationalphilologie und Vergleichende Literaturgeschichte zwischen 1890 und 1910', in L. Danneberg et. al. (eds.), \emph{Stil, Schule, Disziplin}, Frankfurt/M., 335-361.

--- 2006: \emph{The Implied Author}. Berlin.

--- 2008: `Historische Wissenschaften --- Geisteswissenschaften', in S. Haupt \& S. B. W\"urffel (eds.), \emph{Handbuch Fin de Si\`ecle. 1885-1914}, Stuttgart, 662-679.

--- 2011: `Six Ways Not to Save the Implied Author', in \emph{Style} 45, 67-79.

Kindt, W. \& Schmidt, S. J. (eds.) 1976: \emph{Interpretationsanalysen. Argumentationsstrukturen in literaturwissenschaftlichen Interpretationen}, M\"unchen.

Klauk, T. \& K\"oppe T. 2014a: `Telling vs. Showing', in P. H\"uhn (ed.), \emph{Handbook of Narratology}, Berlin.

--- 2014b: `Bausteine einer Theorie der Fiktionalit\"at', in T. Klauk \& T. K\"oppe, In: \emph{Fiktionalit\"at. Ein interdisziplin\"ares Handbuch}, Berlin.

--- 2014c: `On the Very Idea of the Telling vs. Showing Distinction', \emph{Journal of Literary Semantics} 43, 25--42.

--- 2015: `Distance in Fiction', in L. Nordrum \& A. Bareis (eds.), \emph{How to Make Believe: The Fictional Truths of the Representational Arts}, Berlin.

Klauk, T. \& K\"oppe, T. \& Onea, E. 2012: `Internally Focalized Narration -- from a Linguistic Point of View', \emph{Scientific Study of Literature} 2, S. 218--242.

Klauk, T. \& K\"oppe, T. \& Rami, D. (eds.) 2014: \emph{Semantics of Fictional Discourse (Journal of Literary Theory 8)}, Berlin.

K\"oppe, T. 2005: `Prinzipien der Interpretation -- Prinzipien der Rationalit\"at. Oder: Wie erkundet man fiktionale Welten?', \emph{Scientia Poetica} 9, 310-329.

--- 2007: `Vom Wissen in Literatur', \emph{Zeitschrift f\"ur Germanistik} 17, 398-410.

--- 2008a: `Konturen einer analytischen Literaturtheorie', in G. Thuswaldner, \emph{Derrida und danach? Literaturtheoretische Diskurse der Gegenwart}, Wiesbaden.

--- 2008b: \emph{Literatur und Erkenntnis. Studien zur kognitiven Signifikanz fiktionaler literarischer Werke}, Paderborn. 

--- 2009: `Was sind kognitive Kunstfunktionen?', in D. Feige et al. (eds.), \emph{Funktionen von Kunst}, Bern.

--- 2011: `Literatur und Wissen: Zur Strukturierung des Forschungsfeldes und seiner Kontroversen', in T. K\"oppe (ed.), \emph{Literatur und Wissen. Theoretisch-methodische Zug\"ange}, Berlin.

--- 2012: `On Making and Understanding Imaginative Experiences in Our Engagement with Fictional Narratives', in J. Daiber et al. (eds.), \emph{Understanding Fiction. Knowledge and Meaning in Literature}, M\"unster.

--- 2014a: `Fiktionalit\"at in der Neuzeit', in T. Klauk \& T. K\"oppe (eds.), Fiktionalit\"at, Berlin.

--- 2014b: `Institutionelle Theorien der Fiktionalit\"at', in T. Klauk \& T. K\"oppe (eds.), Fiktionalit\"at, Berlin. 

--- 2014c: `Theoretische Rezeptionspsychologie der Fiktionalit\"at', in T. Klauk \& T. K\"oppe (eds.), Fiktionalit\"at, Berlin.

--- 2014d: `Fiktive Tatsachen', in T. Klauk \& T. K\"oppe (eds.), \emph{Fiktionalit\"at}, Berlin.

K\"oppe, T., \& Dennerlein, C. \& Werner, J. 2008: `Interpretation: Struktur und Evaluation in handlungstheoretischer Perspektive', \emph{Journal of Literary Theory} 3, 1-18.

K\"oppe, T. \& Kindt, T. 2011: `Unreliable Narration With a Narrator and Without', \emph{Journal of Literary Theory} 5, 81--93.

--- 2014: \emph{Erz\"ahltheorie. Eine Einf\"uhrung}. Stuttgart.

K\"oppe, T. \& St\"uhring, J. 2011: `Against Pan-Narrator Theories', \emph{Journal of Literary Semantics} 40, 59--80.

--- 2015: `Against Pragmatic Arguments for Pan-Narrator Theories: The Case of Hawthorne's \emph{Rappaccini's Daughter}', in D. Birke \& T. K\"oppe (eds.), \emph{Author and Narrator. Transdisciplinary Contributions to a Narratological Debate,} Berlin.

K\"oppe, T. \& Winko, S. 2010: `Methoden der analytischen Literaturwissenschaft', in A. \&. V. N\"unning (eds.), \emph{Methoden der literatur- und kulturwissenschaftlichen Textanalyse}, Stuttgart.

--- 2011: `Zum Vergleich literaturwissenschaftlicher Interpretationen', in A. Mauz \& H.v.Sass (eds.), \emph{Hermeneutik des Vergleichs}, W\"urzburg.

--- 2013: \emph{Neuere Literaturtheorien}. Stuttgart.

Kolodny, N. 2011: `Aims as Reasons', in J. Wallace et al. (eds.), \emph{Reasons and Recognition}, Oxford.

Kremer, D. 2010: \emph{E.T.A. Hoffmann. Leben -- Werk -- Wirkung}, Stuttgart.

Kripke, S. 2011: `Vacuous Names and Fictional Entities', in S. Kripke, \emph{Philosophical Troubles}, Oxford.

Lamarque, P. 1990: `Reasoning to What Is True in Fiction', \emph{Argumentation} 4, 333-346.

--- 1996: \emph{Fictional Points of View}, Ithaca. 

--- 2009: \emph{The Philosophy of Literature}, Oxford. 

Lamarque, P. \& Olsen, S. 1994: \emph{Truth, Fiction, and Literature}, Oxford.

Landy, J. 2012: \emph{How to Do Things with Fictions}, Oxford.

Levinson, J. 1996: \emph{The Pleasures of Aesthetics}, Ithaca.

Lewis, D. 1978: `Truth in Fiction', \emph{American Philosophical Quarterly} 15, 37-46.

Lindemann, B. 1987: `Einige Fragen an eine Theorie der sprachlichen Perspektivierung', in P. Canisius (ed.), \emph{Perspektivit\"at in Sprache und Text}, Bochum.

Martin, M.G.F. 2002: `The Transparency of Experience', \emph{Mind and Language} 17, 376-425.

Mart\'inez, M. \& Michael Scheffel 1999: \emph{Einf\"uhrung in die Erz\"ahltheorie}, M\"unchen.

Martinich, A. \& Stroll, A. 2007: \emph{Much Ado about Nonexistence, Fiction and Reference}, Lanham.

Martus, S. \& Spoerhase, C. 2009: `Praxeologie der Literaturwissenschaft', in \emph{Geschichte der Germanistik} 35/36, 89-96.

McGinn, C. 2004: \emph{Mindsight}, Cambridge (Mass.).

Mikkonen, Jukka 2013: \emph{The Cognitive Value of Philosophical Fiction}, London.

Nussbaum, M. 1995: \emph{Poetic Justice}, Boston.

Ohmann, R. 1971: `Speech Acts and the Definition of Literature', \emph{Philosophy and Rhetoric} 4, 1-19.

Olsen, S 2004: `Modes of Interpretation and Interpretative Constraints', \emph{British Journal of Aesthetics} 44, 135-148.

Paisley, L. 2005: \emph{Art and Intention}, Oxford.

Pettersson, A. 1993: `On Walton's and Currie's Analyses of Literary Fiction', \emph{Philosophy and Literature} 17, 84-97.

Pink, T. 1996: \emph{The Psychology of Freedom}, Cambridge.

Raz, J. 2003: `The Practice of Value', in J. Wallace (ed.), \emph{The Practice of Value}, Oxford.

---2011: \emph{From Normativity to Responsibility}, Oxford. 

Rorty, R. 2001: \emph{Critical Dialogues}, Oxford.

Savigny, E. von 1976: \emph{Argumentation in der Literaturwissenschaft}, M\"unchen.

Scanlon, T. 2004: `Reasons, a Puzzling Duality?', in J. Wallace et al. (eds.), \emph{Reason and Value}, Oxford.

Schmidt, S. J. 1979: `Bek\"ampfen Sie das h\"a{\ss}liche Laster der Interpretation! Bek\"ampfen Sie das noch h\"a{\ss}lichere Laster der richtigen Interpretation! ', in T. Kindt \& T. K\"oppe (eds.): \emph{Moderne Interpretationstheorien}, G\"ottingen 2008, 194-225.

Sch\"onert, J. (ed.) 2000: Literaturwissenschaft und Wissenschaftsforschung, Stuttgart.

Searle, J. 1975a: `The Logical Status of Fictional Discourse', \emph{New Literary History} 6, 319-32.

--- 1975b: `A Taxonomy of Illocutionary Acts', in K. Gunderson (ed.), \emph{Language, Mind and Knowledge}, Minneapolis.

--- 2002: \emph{Rationality in Action}, Cambridge (MA). 

Shah, N. 2003: `How Truth Governs Belief', \emph{The Philosophical Review} 112, 447-82.

--- 2008: `How Action Governs Intention', \emph{Philosopher's Imprint} 8, 1-19.

Shusterman, R. 1978: `The Logic of Interpretation', \emph{The Philosophical Quarterly} 28, 310-324.

Stock, K. 2011: `Fictive Utterance and Imagining I', \emph{Proceedings of the Aristotelian Society}, Supplementary Volume 85, 145-61.

Stanzel, F. 2008: \emph{Theorie des Erz\"ahlens}, G\"ottingen. 

Stecker, R. 2005: \emph{Aesthetics and the Philosophy of Art}, Lanham.

Stout, J. 1982: `What is the Meaning of a Text?', \emph{New Literary History} 14, 1--12.

Strube, W. 1992: `\"Uber Kriterien der Beurteilung von Textinterpretationen', in L. Danneberg \& F. Vollhardt, \emph{Vom Umgang mit Literatur und Literaturgeschichte}, Stuttgart.

St\"uhring, J. 2011: `Unreliability, Deception, and Fictional Facts', \emph{Journal of Literary Theory} 5, 95--107.

Tepe, P., Rauper, J. \& Semlow, T.: \emph{Interpretationsprobleme am Beispiel von E.T.A. Hoffmanns `Der Sandmann',} W\"urzburg.

Thompson, G./Hunston, S. (eds.) 2000: \emph{Evaluation in Text}, Oxford 2000.

Toulmin, S. 1958: \emph{The Uses of Argument}, Cambridge.

Velleman, D. 2000a: `The Guise of the Good', in D. Velleman, \emph{The Possibility of Practical Reason}, Oxford.

--- 2000b: `On the Aim of Belief', in D. Velleman, \emph{The Possibility of Practical Reason}, Oxford.

Walsh, R. 2007: \emph{The Rhetoric of Fictionality}, Columbus.

Walton, K. 1970: `Categories of Art', \emph{The Philosophical Review} 79, 334-367.

--- 1990: \emph{Mimesis as Make-Believe}, Harvard.

--- 1993: `How Marvelous! Toward a Theory of Aesthetic Value', \emph{Journal of Aesthetics and Art Criticism} 51, 499-510.

--- 2013: `Fictionality and Imagination Reconsidered', in C. Barbero et al. (eds.), \emph{From Fictionalism to Realism}, Newcastle.

Weitz, M. 1964: \emph{Hamlet and the Philosophy of Literary Criticism}, Chicago.

Winko, S. 2002: `Autor-Funktionen. Zur argumentativen Verwendung von Autorkonzepten in der gegenw\"artigen literaturwissenschaftlichen Interpretationspraxis', in H. Detering (ed.), \emph{Autorschaft. Positionen und Revisionen}, Stuttgart, 334-354.

--- 2015a: `Zur Plausibilit\"at als Beurteilungskriterium literaturwissenschaftlicher Interpretationen', in A. Albrecht et al. (eds.), \emph{Theorien, Methoden und Praktiken des Interpretierens}, Berlin, 483-511.

--- 2015b: `Standards literaturwissenschaftlichen Argumentierens, Grundlagen und Forschungsfragen', in \emph{Germanisch-Romanischen Monatsschrift} 65, 13-29.

Wolterstorff, N. 1980: \emph{Works and Worlds of Art}, Oxford.

Worthmann, F. 2013: `Wie analysiert man literarische Wertungen? ', in G. Rippl \& S. Winko (eds.), \emph{Handbuch Kanon und Wertung}, Stuttgart, 402-407.

Zabka, T. 2005: \emph{Pragmatik der Literaturinterpretation}, T\"ubingen.

Zipfel, F. 2001: \emph{Fiktion, Fiktivit\"at, Fiktionalit\"at}, Berlin.

\end{hangparas}
