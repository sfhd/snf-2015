\vspace{.2cm}
\noindent\textbf{\large Bibliography}
\vspace{.2cm}

\begin{hangparas}{.5cm}{1}

Bareis, J. 2008: \emph{Fiktionales Erz\"ahlen}, G\"oteborg.

--- 2009: `Was ist wahr in der Fiktion?', \emph{Scientia Poetica} 13, 230-254.  

Budd, M. 1995: \emph{Values of Art. Pictures, Poetry and Music}, London. 

Burge, T. 2010: \emph{Origins of Objectivity}, Oxford.

--- 2013a: `Content Preservation', in T. Burge, \emph{Cognition Through Understanding}, Oxford.

--- 2013b: `Postscript to `Content Preservation'', in T. Burge, \emph{Cognition Through Understanding}, Oxford.

Carroll, N. 1997: `Fiction, Non-Fiction, and the Film of Presumptive Assertion: A Conceptual Analysis', in R. Allen \& M. Smith (eds.), \emph{Film Theory and Philosophy}, Oxford.

Choi, J. 2005: `Leaving It Up to the Imagination. POV Shots and Imagining from the Inside', \emph{The Journal of Aesthetics and Art Criticism} 63, 17--25. 

Currie, G. 1990: \emph{The Nature of Fiction}, Cambridge. 

--- 1991: `Interpreting Fictions', in R. Freadman and L. Reinhardt (eds.), \emph{On Literary Theory and Philosophy}, Basingstoke. 

--- 2003: `Interpretation in Art', in J. Levinson (ed.), \emph{The Oxford Handbook of Aesthetics}, Oxford.

Davies, D. 2007: \emph{Aesthetics and Literature}, London.

Diehl, N. 2009: `Imaginging De Re and the Symmetry Thesis of Narration', \emph{The Journal of Aesthetics and Art Criticism} 67, 15--24.

Eco, U. 1999: \emph{Die Grenzen der Interpretation}, M\"unchen. 

Evans, G. 1982: \emph{The Varieties of Reference}, Oxford. 

Everett, A. 2013: \emph{The Nonexistent}, Oxford. 

F\o{}llesdal, D. \& Wall\O{}e L. \& Elster, J. 2008: `Die hypothetisch-deduktive Methode in der Literaturinterpretation', in T. Kindt \& T. K\"oppe (eds.), \emph{Moderne Interpretationstheorien}, G\"ottingen.

Genette, G. 1980: \emph{Narrative Discourse}, New York. 

Gertken, J. \& K\"oppe, T. 2009: `Fiktionalit\"at', in F. Jannidis et al. (eds.), \emph{Grenzen der Literatur. Zu Begriff und Ph\"anomen des Literarischen}, Berlin. 

Goldman, A. 1995: \emph{Aesthetic Value}, Boulder. 

--- 2006: `The Experiential Account of Aesthetic Value', \emph{The Journal of Aesthetics and Art Criticism} 64, 333-342.

Habermas, T. 2006: `Who speaks? Who looks? Who feels? Point of view in autobiographical narratives', \emph{International Journal of Psychoanalysis} 87, 497-518. 

Hieronymi, P. 2014: `Reflection and Responsibility', \emph{Philosophy and Public Affairs} 42, 3-41.

Kania, A. 2005: `Against the Ubiquity of Fictional Narrators', \emph{The Journal of Aesthetics and Art Criticism} 63, 47--54.

Kindt, T. 2008: \emph{Unzuverl\"assiges Erz\"ahlen und literarische Moderne. Eine Untersuchung der Romane von Ernst Wei{\ss}}, T\"ubingen. 

--- *

Kindt, T. \& M\"uller H. *2003a: `Narrative Theory and/or/as Theory of Interpretation', in T. Kindt \& H. M\"uller (eds.), \emph{What Is Narratology? Questions and Answers Regarding the Status of a Theory}, Berlin.

Klauk, T. \& K\"oppe T. 2014a: `Telling vs. Showing', in P. H\"uhn (ed.), \emph{Handbook of Narratology}, Berlin.

--- 2014b: `Bausteine einer Theorie der Fiktionalit\"at', in T. Klauk \& T. K\"oppe, In: \emph{Fiktionalit\"at. Ein interdisziplin\"ares Handbuch}, Berlin.

--- *2014: `On the Very Idea of the Telling vs. Showing Distinction', \emph{Journal of Literary Semantics} 43, 25--42.

--- 2015: `Distance in Fiction', in T. Klauk \& T. K\"oppe (eds.), \emph{How to Make Believe: The Fictional Truths of the Representational Arts}, Berlin. \textbf{\emph{(* In press}).}

Klauk, T. \& K\"oppe, T. \& Onea, E. 2012: `Internally Focalized Narration -- from a Linguistic Point of View', \emph{Scientific Study of Literature} 2, S. 218--242.

Klauk, T. \& K\"oppe, T. \& Rami, D. (eds.) 2014: \emph{Semantics of Fictional Discourse (Journal of Literary Theory 8)}, Berlin.

K\"oppe, T. 2005: `Prinzipien der Interpretation -- Prinzipien der Rationalit\"at. Oder: Wie erkundet man fiktionale Welten?', \emph{Scientia Poetica} 9, 310-329.

--- 2007: `Vom Wissen in Literatur', \emph{Zeitschrift f\"ur Germanistik} 17, 398-410.

--- *2008: \emph{Literatur und Erkenntnis. Studien zur kognitiven Signifikanz fiktionaler literarischer Werke}, Paderborn. 

--- 2008a: `Konturen einer analytischen Literaturtheorie', in G. Thuswaldner, \emph{Derrida und danach? Literaturtheoretische Diskurse der Gegenwart}, Wiesbaden.

--- *2009: `Was sind kognitive Kunstfunktionen?', in D. Feige et al. (eds.), \emph{Funktionen von Kunst}, Bern.

--- 2009a: `Fiktion, Praxis, Spiel. Was leistet der Spielbegriff bei der Kl\"arung des Fiktionalit\"atsbegriffs?', in T. Anz \& H. Kaulen (eds.), \emph{Literatur als Spiel.  Evolutionsbiologische, \"asthetische und p\"adagogische Konzepte}, Berlin.

--- 2011: `Literatur und Wissen: Zur Strukturierung des Forschungsfeldes und seiner Kontroversen', in T. K\"oppe (ed.), \emph{Literatur und Wissen. Theoretisch-methodische Zug\"ange}, Berlin.

--- *2012: `On Making and Understanding Imaginative Experiences in Our Engagement with Fictional Narratives', in J. Daiber et al. (eds.), \emph{Understanding Fiction. Knowledge and Meaning in Literature}, M\"unster.

--- 2012a: `E.D. Hirschs Auseinandersetzung mit Gadamer', in C. Dutt (ed.), \emph{Gadamers philosophische Hermeneutik und die Literaturwissenschaft}, Heidelberg.

--- *2014: `Fiktive Tatsachen', in T. Klauk \& T. K\"oppe (eds.), \emph{Fiktionalit\"at}, Berlin.

--- 2014a: `Fiktionalit\"at in der Neuzeit', in T. Klauk \& T. K\"oppe (eds.), Fiktionalit\"at, Berlin.

--- 2014b: `Institutionelle Theorien der Fiktionalit\"at', in T. Klauk \& T. K\"oppe (eds.), Fiktionalit\"at, Berlin. 

--- 2014c: `Theoretische Rezeptionspsychologie der Fiktionalit\"at', in T. Klauk \& T. K\"oppe (eds.), Fiktionalit\"at, Berlin.

K\"oppe, T., \& Dennerlein, C. \& Werner, J. *2008a: `Interpretation: Struktur und Evaluation in handlungstheoretischer Perspektive', \emph{Journal of Literary Theory} 3, 1-18.

K\"oppe, T. \& Kindt, T. 2011: `Unreliable Narration With a Narrator and Without', \emph{Journal of Literary Theory} 5, 81--93.

--- 2014: \emph{Erz\"ahltheorie. Eine Einf\"uhrung}. Stuttgart.

K\"oppe, T. \& St\"uhring, J. 2011: `Against Pan-Narrator Theories', \emph{Journal of Literary Semantics} 40, 59--80.

--- 2015: `Against Pragmatic Arguments for Pan-Narrator Theories: The Case of Hawthorne's \emph{Rappaccini's Daughter}', in D. Birke \& T. K\"oppe (eds.), \emph{Author and Narrator. Transdisciplinary Contributions to a Narratological Debate,} Berlin.

K\"oppe, T. \& Winko, S. 2010: `Methoden der analytischen Literaturwissenschaft', in A. \&. V. N\"unning (eds.), \emph{Methoden der literatur- und kulturwissenschaftlichen Textanalyse}, Stuttgart.

--- 2011: `Zum Vergleich literaturwissenschaftlicher Interpretationen', in A. Mauz \& H.v.Sass (eds.), \emph{Hermeneutik des Vergleichs}, W\"urzburg.

--- 2013: \emph{Neuere Literaturtheorien}. Stuttgart.

Kolodny, N. 2011: `Aims as Reasons', in J. Wallace et al. (eds.), \emph{Reasons and Recognition}, Oxford.

Kripke, S. 2011: `Vacuous Names and Fictional Entities', in S. Kripke, \emph{Philosophical Troubles}, Oxford.

Lamarque, P. 1990: `Reasoning to What Is True in Fiction', \emph{Argumentation} 4, 333-346.

--- 1996: \emph{Fictional Points of View}, Ithaca. 

--- 2009: \emph{The Philosophy of Literature}, Oxford. 

Lamarque, P. \& Olsen, S. 1994: \emph{Truth, Fiction, and Literature}, Oxford.

Lewis, D. 1978: `Truth in Fiction', \emph{American Philosophical Quarterly} 15, 37-46.

Lindemann, B. 1987: `Einige Fragen an eine Theorie der sprachlichen Perspektivierung', in P. Canisius (ed.), \emph{Perspektivit\"at in Sprache und Text}, Bochum.

Mart\'inez, M. \& Michael Scheffel 1999: \emph{Einf\"uhrung in die Erz\"ahltheorie}, M\"unchen.

Martinich, A. \& Stroll, A. 2007: \emph{Much Ado about Nonexistence, Fiction and Reference}, Lanham.

McHugh, C. 2014: `Exercising Doxastic Freedom', \emph{Philosophy and Phenomenological Research} 88, 1-37.

Mikkonen, Jukka 2013: \emph{The Cognitive Value of Philosophical Fiction}, London.

Ohmann, R. 1971: `Speech Acts and the Definition of Literature', \emph{Philosophy and Rhetoric} 4, 1-19.

Olsen, S 2004: `Modes of Interpretation and Interpretative Constraints', \emph{British Journal of Aesthetics} 44, 135-148.

Owens, D. 2006: `Testimony and Assertion', \emph{Philosophical Studies} 130, 105--29.

Paisley, L. 2005: \emph{Art and Intention}, Oxford.

Pettersson, A. 1993: `On Walton's and Currie's Analyses of Literary Fiction', \emph{Philosophy and Literature} 17, 84-97.

Raz, J. 2003: `The Practice of Value', in J. Wallace (ed.), \emph{The Practice of Value}, Oxford.

---2011: \emph{From Normativity to Responsibility}, Oxford. 

Scanlon, T. 2004: `Reasons, a Puzzling Duality?', in J. Wallace et al. (eds.), \emph{Reason and Value}, Oxford.

Schroeder, M. 2012: `Value Theory', \emph{The Stanford Encyclopedia of Philosophy}, \textless{}http://plato.stanford.edu\textgreater{}.

Searle, J. 1975a: `The Logical Status of Fictional Discourse', \emph{New Literary History} 6, 319-32.

--- 1975b: `A Taxonomy of Illocutionary Acts', in K. Gunderson (ed.), \emph{Language, Mind and Knowledge}, Minneapolis.

--- 2002: \emph{Rationality in Action}, Cambridge (MA). 

Shah, N. 2003: `How Truth Governs Belief', \emph{The Philosophical Review} 112, 447-82.

--- 2008: `How Action Governs Intention', \emph{Philosopher's Imprint} 8, 1-19.

Shpall, S. \& Wilson, G. 2012: `Action', \emph{The Stanford Encyclopedia of Philosophy}. \textless{}http://plato.stanford.edu\textgreater{}.

Shusterman, R. 1978: `The Logic of Interpretation', \emph{The Philosophical Quarterly} 28, 310-324.

Stock, K. 2011: `Fictive Utterance and Imagining I', \emph{Proceedings of the Aristotelian Society}, Supplementary Volume 85, 145-61.

--- *

Stanzel, F. 2008: \emph{Theorie des Erz\"ahlens}, G\"ottingen. 

Stecker, R. 2005: \emph{Aesthetics and the Philosophy of Art}, Lanham.

Stout, J. 1982: `What is the Meaning of a Text?', \emph{New Literary History} 14, 1--12.

Strube, W. 1992: `\"Uber Kriterien der Beurteilung von Textinterpretationen', in L. Danneberg \& F. Vollhardt, \emph{Vom Umgang mit Literatur und Literaturgeschichte}, Stuttgart.

St\"uhring, J. 2011: `Unreliability, Deception, and Fictional Facts', \emph{Journal of Literary Theory} 5, 95--107.

Velleman, D. 2000a: `The Guise of the Good', in D. Velleman, \emph{The Possibility of Practical Reason}, Oxford.

--- 2000b: `On the Aim of Belief', in D. Velleman, \emph{The Possibility of Practical Reason}, Oxford.

Walsh, R. 2007: \emph{The Rhetoric of Fictionality}, Columbus.

Walton, K. 1970: `Categories of Art', \emph{The Philosophical Review} 79, 334-367.

--- 1990: \emph{Mimesis as Make-Believe}, Harvard.

--- 2013: `Fictionality and Imagination Reconsidered', in C. Barbero et al. (eds.), \emph{From Fictionalism to Realism}, Newcastle.

Wolterstorff, N. 1980: \emph{Works and Worlds of Art}, Oxford.

Zipfel, F. 2001: \emph{Fiktion, Fiktivit\"at, Fiktionalit\"at}, Berlin.

\end{hangparas}
