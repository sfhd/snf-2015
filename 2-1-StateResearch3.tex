

\vspace{.2cm}
\noindent\textbf{Subproject 3: Foundational Issues (Philosophy / Fribourg)}
\vspace{.2cm}


\vspace{.2cm}
\noindent On the views which say that fictional texts invite or prescribe us to imagine certain things (e.g. what makes up their content), the utterances constituting a work of fiction, and the contents they express, are more than mere psychological triggers of our imagination. They also provide normative guidance for, and put normative constraints, on the imaginative responses we give, determining which of them are correct and which are not. Much philosophical work in this area is devoted to characterizing what it is that guides and constrains our imaginative responses. Thus, many have offered a systematic account of (the principles that determine) the content of a work of fiction (see e.g. Lewis 19\textbf{78}; Currie 1990, \textbf{ch. 2}; Walton 1990, \textbf{ch.?}; Lamarque \& Olsen 1994, \textbf{ch. 4}; Davies 2007, ch. 4) and of the kind of acts or utterances by which that content is expressed (Searle 1975a; Currie 1990, \textbf{ch. 1}; Lamarque and Olsen 1994, \textbf{ch. 2}; Davies 2007, ch. 3). But the underlying question of how utterances of that kind --- with the sort of content they are presumed to have --- could be apt to provide any normative guidance for imagination, and how various forms of imaginative thought (and possibly also experience) could be subject to normative directions and constraints in the first place, has received much less attention. Our research aims to address these foundational issues by elucidating the structural features that are presupposed by the idea that fictional utterances can serve as normative guides for imaginative thought.

\vspace{.2cm}
\noindent\textbf{\emph{The Seeming Analogy with Reports and Beliefs.}} A natural way of approaching the issues is suggested by how fictional utterances and the corresponding imaginative responses are usually characterized. What someone who tells a fictional story like a fairy tale, say, does is often described as resembling, or even being parasitic on, what is done in a factual report. More specifically, the person telling the fictional story is taken, not to report, but merely to pretend or make as if to report (Searle 1975a; Lewis 1978; Kripke 2011), to imitate reporting (Ohmann 1971), or to do what very much looks like reporting but is done with a different, distinctly fictive intent (Currie 1990\textbf{, ch. 1}; Lamarque \& Olsen 1994, ch. 2; Davies 2007, ch. 3). Similarly, what the reader or hearer of a fictional story does when she adopts the `fictive stance' and forms the relevant imaginative responses is often characterized as something that is closely connected to belief. In particular, she is sometimes taken not to believe, but instead to make-believe that what she is being told is the case (Currie 1990; Walton 1990), or to make-believe that it is being told with the standard communicative commitments in force (Lamarque \& Olsen 1994, ch. 2). Alternatively, she may be said to entertain contents as unasserted, rather than as asserted like in the case of belief \textbf{(Carroll 1997).}

Given these similarities in input and output, it is tempting to assume a similar parallel between the two forms of communication with respect to their underlying normative structure. Accordingly, we are supposed to imagine what is presented by a fictional story as much as as we are supposed to believe what is represented by a factual report. That is, the fictional utterances constituting a fictional story may be said to normatively guide and constrain what we imagine in a similar fashion as the assertive utterances constituting a factual report normatively guide and constrain what we believe. Indeed, the accounts that are offered in the fictional case are usually framed in the same terms as those given in the non-fictional case, namely in terms of speakers' communicative intentions and commitments (Currie 1990; Carroll 1997; Davies 2007), discourse-specific conversational rules and practices (Walton 1990), or combinations thereof (Searle 1975a; Lamarque \& Olsen 1994). It is generally accepted that the intentions, commitments or rules that govern the non-fictional case are missing or somehow suspended in the fictional case. What is striking, though, is that the accounts remain remarkably unspecific when it comes to positively explaining how intentions, commitments or rules can endow fictional utterances with the normative authority over imaginative thoughts that they are supposed to have.

(\textbf{* Too long. *)} More precisely, fictional utterances are commonly said to lack the commitment to truth that is characteristic of assertive utterances (Ohmann 1971; Searle 1975a), just as imaginative thoughts are assumed to lack the sensitivity to truth that is exhibited by beliefs (Velleman 2000b; Searle 2002, ch. 2; Burge 2010, ch. 8). This shared indifference towards truth might be expected to explain why fictional utterances can serve as guides to imagination, just as the shared connection to truth helps to explain why assertions are apt to guide beliefs. But this expectation is misguided. We have a rough idea of the sense in which genuine assertions are said to guide or constrain what we believe: beliefs are cognitive states whose function it is to be true and which are sensitive to evidence in favor of (or against) their truth (Velleman 2000a; 2000b; Shah 2003; Burge 2010, ch. 8; McHugh 2012), while assertions are utterances that involve a commitment to truth and thus seem to provide evidence for the truth of, and hence for believing, what is asserted (cf. Burge 2013a; 2013b; \textbf{Owens 2006}). In the fictional case, by contrast, the appeal to truth only serves to indicate which fundamental features fictional utterances and imaginative thoughts do \emph{not} possess, and it is difficult to see how the absence of these features could account for the normative relation that is supposed to hold between utterance and response: * That fictional utterances lack a commitment to truth only suggests that they lack the normative force that genuine assertions possess, not that they have any normative force of their own; and the observation that imaginative thoughts are insensitive to truth suggests only that they are insensitive to the evidential considerations that guide beliefs, not that they are sensitive to any other kind of normative considerations at all.

In their attempt to offer a more positive characterization of the normativity of fiction-based imagining, some accounts appeal to the notion of a fictional practice that is regulated by specific communicative rules according to which we are required to respond to fictional utterances by forming corresponding imaginative thoughts (Searle 1975a; Walton 1990, \textbf{ch.?}; Lamarque and Olsen 1994, ch. 2). But this idea is hardly any progress. For saying that there are rules that require us to imagine certain things in response to fictional utterances does not amount to more than saying that fictional utterances have normative authority with regard to our imaginative thoughts. What we are trying to understand is why, or in virtue of what, these rules apply and what makes them normatively binding in the first place.

The most popular idea in this context is the assumption that fictional utterances are made with a specific communicative intent, namely the intention to get the reader or hearer to form certain imaginative thoughts, rather than to adopt certain beliefs as in the case of assertion (Currie 1990, ch. 1; Lamarque and Olsen 1994, ch. 2; Davies 2007, ch. 3; Stock 2011; Mikkonen 2013, ch. 2). Yet it is unclear how an intention like that could have any normative impact on its own. Intentions in themselves do not constitute normative reasons, and so there is nothing wrong or incorrect about not carrying them out (cf. Raz 2011, ch. 8; Kolodny 2014; or Scanlon 2004). In the case of assertions, it is not the intention that is of normative significance. As already indicated, what is significant is rather the fact that the intended beliefs are cognitive states which are sensitive to truth-supporting evidence, and the supposition that the assertions themselves provide such evidence (however weak or defeasible it may be). In the case of fictional utterances, by contrast, we merely know that these fundamental connections to truth and evidence are missing, which brings us back to the problem that we lack a positive account of the corresponding feature(s) that produce the normative force instead.

\vspace{.2cm}
\noindent\textbf{\emph{The Second Normative Relation}.} In contrast to the normativity of fiction-based imagining, the normativity of appreciative responses to fictional texts is relatively unproblematic. The main reason for this is that we already have a fairly good grasp of how aesthetic appreciation works (Goldman, Budd, *) and, in particular, of the fact that aesthetic judgements are structurally similar to perceptual or descriptive judgements in that both are experience- or evidence-based (* Sibley, McDowell, Goldman, Schellekens, Dorsch). Whenever the fact that a given fictional text invites a certain imaginative response gives us a prima facie reason to ascribe a certain aesthetic value to the text, this happens because --- and to the extent to which --- the fact contributes to the text's aesthetic value and thus ensures that our matching aesthetic evaluation is appropriate. For example, we are entitled to treat X as a great masterpiece partly because \ldots helps to realize the aesthetic greatness of X. Of course, there is the factual issue of why certain prescriptions to imagine are merit-constituting properties, and others are not (see subproject \%). But this issue has no bearing on the more general truth that our recognition of the aesthetic qualities of a fictional text -- which typically includes its property of inviting us to imagine certain things --- justifies us in ascribing a certain aesthetic value to the text precisely because the qualities partly constitute the value and thus render its ascription likely to be correct.

One might still doubt that imaginative responses could rationally ground aesthetic evaluations since, while imagining cannot provide support for the adoption of a judgement-like attitude (e.g. imagining \emph{p} and \emph{if p, then q} may license us only to imagine \emph{p}, but not to judge \emph{q}), aesthetic evaluation does involve such an attitude (Dorsch 2015). But this worry is misguided, given that our aesthetic assessment of a fictional work is not the result of reasoning the premisses of which comprise what we imagine. Rather, our appreciation is grounded in a special kind of feeling or sentiment in response to our partly imaginative experience of the work (Hume; Kant; Budd; Goldman; Walton; Levinson *); or, alternatively, in reasoning that starts off with considerations about the work's aesthetic qualities, among them facts about what the text invites us to imagine (Bender *; Dorsch 2013).

Accordingly, the foundational part of the research project focuses largely on the issue of how to best account for the first normative relation, as well as the issue of which implications such an account has for our understanding of the link between the two normative relations (e.g. whether one can be derived from the other).

