

\vspace{.2cm}
\noindent\textbf{Subproject 3: Foundational Issues (Dorsch/Philosophy/Fribourg)}
\vspace{.2cm}


\vspace{.2cm}
\noindent \textbf{\emph{(NR1).}} Much recent work in the philosophy of fiction is devoted to characterizing what is supposed to guide and constrain our imaginative responses to fictional works --- that is, the accounts focus on the principles that determine the \emph{content} of a work of fiction (see e.g. Lewis 19\textbf{78}; Currie 1990, \textbf{ch. 2}; Walton 1990, \textbf{ch. 4}) and the kind of \emph{utterances} by which that content is expressed (Searle 1975a; Currie 1990, \textbf{ch. 1}; Lamarque/Olsen 1994, \textbf{ch. 2}; Davies 2007, ch. 3). But the underlying question of how utterances of that kind --- with the sort of content they are presumed to have --- could be apt to provide any normative guidance for imagination, and how various forms of imaginative thought (and possibly also experience) could be subject to normative directions and constraints in the first place, has received much less attention. Our research aims to address these foundational issues by elucidating the structural features that are presupposed by the idea that fictional utterances can serve as normative guides for imaginative thought.

\vspace{.2cm}
\noindent\textbf{\emph{The (Dis)Analogy with Reports and Beliefs.}} A natural way of approaching the issues is suggested by how fictional utterances and the corresponding imaginative responses are usually characterized. What someone who tells a fictional story does is often described as resembling, or even being parasitic on, what is done in a factual report: the person telling the fictional story is taken, not to report, but merely to pretend or make as if to report (Searle 1975a; Lewis 1978; Kripke 2011), to imitate reporting (Ohmann 1971), or to do what very much looks like reporting but is done with a different, distinctly fictive intent (Currie 1990\textbf{, ch. 1}; Lamarque/Olsen 1994, ch. 2; Davies 2007, ch. 3). Similarly, what the reader or hearer of a fictional story does when she adopts the 'fictive stance' and forms the relevant imaginative responses is often characterized as something that is closely connected to belief: she is taken not to believe, but instead to make-believe that what she is being told is the case (Currie 1990; Walton 1990), or that it is being told with the standard communicative commitments in force (Lamarque/Olsen 1994, ch. 2). Accordingly, the way how our imaginative responses are guided and constrained by the fictional utterances in a story is usually accounted for in the same terms as the way our doxastic responses are guided and constrained by the assertve utterances in a report, namely in terms of speakers' communicative intentions and commitments (Currie 1990; Carroll 1997; Davies 2007), discourse-specific conversational rules and practices (Walton 1990), or combinations thereof (Searle 1975a; Lamarque/Olsen 1994). It is generally accepted that the intentions, commitments or rules that govern the non-fictional case are missing or somehow suspended in the fictional case. What is striking, though, is that the accounts remain remarkably unspecific when it comes to positively explaining how intentions, commitments or rules can endow fictional utterances with the normative authority over imaginative thoughts that they are supposed to have.


\vspace{.2cm}
\noindent \textbf{\emph{Communicative Commitments, Rules, and Intentions.}} More precisely, that fictional utterances lack the commitment to truth that is characteristic of assertive utterances (Ohmann 1971; Searle 1975a), and that imaginative thoughts lack the sensitivity to truth that is exhibited by beliefs (Velleman 2000b; Searle 2002, ch. 2; Burge 2010, ch. 8), does not imply that fictional utterances can serve as guides to imagination in the sense assertions are apt to guide beliefs. We have a rough idea of the sense in which genuine assertions are said to guide or constrain what we believe: beliefs are cognitive states whose function it is to be true and which are sensitive to evidence in favor of (or against) their truth (Velleman 2000a; 2000b; Shah 2003; Burge 2010, ch. 8; McHugh 2014), while assertions are utterances that involve a commitment to truth and thus seem to provide evidence for the truth of, and hence for believing, what is asserted (cf. Burge 2013a; 2013b; Owens 2006). In the fictional case, by contrast, the appeal to truth only serves to indicate which fundamental features fictional utterances and imaginative thoughts do \emph{not} possess, and it is difficult to see how the absence of these features could account for the normative relation that is supposed to hold between utterance and response. That fictional utterances lack a commitment to truth only suggests that they lack the normative force that genuine assertions possess, not that they have any normative force of their own; and the observation that imaginative thoughts are insensitive to truth suggests only that they are insensitive to the evidential considerations that guide beliefs, not that they are sensitive to any other kind of normative considerations at all.

In order to offer a more positive characterization, some accounts appeal to the notion of a fictional practice, as regulated by specific communicative rules, according to which we are required to respond to fictional utterances by forming corresponding imaginative thoughts (Searle 1975a; Walton 1990, ch.4; Lamarque/Olsen 1994, ch. 2). But this idea is hardly any progress. For saying that there are rules that require us to imagine certain things in response to fictional utterances does not amount to more than saying that fictional utterances have normative authority with regard to our imaginative thoughts. What we are trying to understand is why, or in virtue of what, these rules apply and what makes them normatively binding in the first place.

The most popular idea in this context is the assumption that fictional utterances are made with a specific communicative intent, namely the intention to get the reader or hearer to form certain imaginative thoughts, rather than to adopt certain beliefs as in the case of assertion (Currie 1990, ch. 1; Lamarque/Olsen 1994, ch. 2; Davies 2007, ch. 3; Stock 2011; Mikkonen 2013, ch. 2). Yet it is unclear how an intention like that could have any normative impact on its own. Intentions in themselves do not constitute normative reasons, and so there is nothing wrong or incorrect about not carrying them out (cf. Raz 2011, ch. 8; Kolodny 2014; or Scanlon 2004).

\vspace{.2cm}
\noindent\textbf{\emph{(NR2).}} In contrast to the normativity of fiction-based imagining, the normativity of appreciative responses to fictional texts is relatively unproblematic. The main reason for this is that we already have a fairly good grasp of how aesthetic appreciation works and, in particular, of the fact that aesthetic judgements are structurally similar to perceptual or descriptive judgements in that both are experience- or evidence-based (Walton 1993; Budd 1995; Goldman 1995; Levinson 1996; Dorsch 2000). Whenever the fact that a given fictional text invites a certain imaginative response gives us a prima facie reason to ascribe a certain aesthetic value to the text, this happens because --- and to the extent to which --- the fact contributes to the text's aesthetic value and thus ensures that our matching aesthetic evaluation is appropriate.

