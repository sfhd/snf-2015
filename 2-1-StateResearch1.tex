\noindent\textbf{\large 2.  Research Plan}

\noindent\textbf{2.1.  State of Research}

\noindent The three subprojects approach the normative relations (NR1) and (NR2) from three complementary angles. Whereas the first subproject studies the factual issue of which specific norms are \emph{de facto} in play in interpretative practice, the second asks the normative question of which specific norms \emph{should} be followed, while the third addresses the foundational issue of where these norms derive their authority from.

\vspace{.2cm}
\noindent\textbf{Subproject 1} explores the interrelation between imagination and appreciation in the practice of literary studies. By examining argumentative structures and evaluative components in a comprehensive sample of `expert interpretations' of a certain literary work (i.e. E.T.A. Hoffmann's `Der Sandmann'), the subproject aims to gain empirical evidence for answering the following general question concerning the normative dimension of `doing interpretation' in literary criticism:

\vspace{-.1cm}
\begin{itemize}[leftmargin=2cm]
\item[\textbf{(Q1)}] Which specific instances of (NR1) and (NR2) are \emph{de facto} accepted by literary scholars as governing our imaginative and appreciative engagement with fictional texts?
\end{itemize}
\vspace{-.1cm}

\noindent Subproject 1 puts a particular emphasis on the specific instances of (NR1) and (NR2) that are examined by Subproject 2.

\vspace{.2cm}
\noindent\textbf{Subproject 2} takes the philosophical debate over what particular works of fiction prescribe us to imagine as a starting point and analyzes arguments to the effect that matters of appreciation should influence these prescriptions. In addition, the project explores the extend to which these prescriptions can be said to guide not only the content of our imaginings but also the way of imagining. Evidence for the importance of these content-transcending aspects of imagining for literary appreciation is taken from the narratological literature on typical narrative features of literary fictions:

\vspace{-.1cm}
\begin{itemize}[leftmargin=2cm]
\item[\textbf{(Q2)}] Which specific instances of (NR1) and (NR2) \emph{should} (or \emph{should not}) govern our imaginative and appreciative engagement with fictional texts?
\end{itemize}
\vspace{-.1cm}

\noindent The results of Subproject 2 will be cross-validated by drawing on the actual interpretative practice as uncovered by Subproject 1. Moreover, Subproject 2 will suggest reconsiderations of the very notion of prescriptions to imagine that need to be assessed in close cooperation with Subproject 3.

\vspace{.2cm}
\noindent\textbf{Subproject 3} aims to investigate why the normative relations between fiction, imagination and appreciation hold and what makes them possible. That is, it tries to identify the source of their normative authority. Concerning (NR2), there are already well-developed views on the normativity of aesthetic appreciation and thus also on how fiction-related prescriptions to imagine can constitute aesthetic qualities that are normatively relevant for appreciation (see below). For this reason, the subproject focuses primarily on the much more neglected normativity of imagining and, especially, on (NR1): 

\vspace{-.1cm}
\begin{itemize}[leftmargin=2cm]
\item[\textbf{(Q3)}] What is the source of the normative authority of the specific instances of (NR1), in virtue of which fictional texts direct our imagination?
\end{itemize}
\vspace{-.1cm}

\noindent Tackling this question will, inter alia, help us to find solutions to problems raised by the Subprojects 1 and 2 --- for instance, to the puzzle of (apparently) conflicting prescriptions to imagine with respect to a particular work. --- While Subproject 1 is intended for a PhD student, Subprojects 2 and 3 are designed for postdocs. All three position will be advertised internationally.

\vspace{.4cm}
\noindent\textbf{Subproject 1: Factual Issues (Kindt/Literary Studies/Fribourg/PhD)}
\vspace{.2cm}

\noindent \textbf{\emph{The Emergence of `Practice Analysis'.}} Inspired by the so-called `practice turn' in science studies and the history of science, literary criticism has --- in the course of the last two decades --- developed a steadily growing interest in illuminating its own practices (cf. e.g. Sch\"onert 2000; Martus/Spoerhase 2009; Albrecht et al. 2015). On the one hand, this trend has resulted in sociologically oriented investigations of different spheres of activity in literary studies. On the other hand, it has revived the very endeavor that the subproject attempts to carry forward, namely the endeavor of an empirical analysis of the practice of interpretation guided by philosophical and linguistical theories of speech acts and argumentation. 

This manner of `practice analysis' originated in the 1970s with a series of studies on `expert interpretations' of literature that, for two reasons, are still an important point of reference for meta-critical investigations like the projected one. First, these studies have developed convincing basic accounts of the language and the speech acts used in literary criticism and, by doing so, of the different operations involved in the activity of scholarly interpretation (cf. Weitz 1964; Beardsley 1970; Fricke 1977). Second, the research in question has provided us with systematic models for analysis and with empirical insights into the argumentative structures of interpretations in literary studies (cf. Grewendorf 1975; Beetz/Meggle 1976; Kindt/Schmidt 1976; von Savigny 1978). 

\vspace{.2cm}
\noindent \textbf{\emph{Relevant Recent Developments.}} In the line of the meta-critical studies of the 1970s, the last years have seen several approaches in literary studies to attain a refined analytical and empirical picture of the discipline's interpretive practices. Three strands of these approaches are of particular relevance for this subproject. 

First, based on Toulmin's understanding of argumentation (Toulmin 1958), Winko has conceived of and compellingly exemplified a model for examining argumentative structures and procedures in scholarly interpretations that allows for the specifics of these structures and procedures in the field of literary studies (cf. Winko 2002, 2015a, 2015b).

Second, stimulated by speech-act- and action-theory, increased attempts have been made recently to advance the project of a pragmatics of interpretation that characterizes the practice of interpretation in literary criticism as an interplay of various basic critical operations related to specific aims, types of claims and conditions of success (cf. Zabka 2005; Dennerlein/K\"oppe/Werner 2008; Kindt 2015a). 

Third, one of the basic operations involved in scholarly interpretation, the act of evaluation, has found particular attention in newer approaches to illuminating elements and compositions of critical discourse: From the perspective of linguistic discourse analysis, Thompson and Hunston have analyzed evaluation as a `stance taking' that is at hand once someone verbally expresses any kind of attitude towards an object (cf. Thompson/Hunston 2000). Following speech act-theoretical conceptions, Winko and von Heydebrand have proposed a more restrictive notion of evaluation; they have systematically explicated the operation as the usage of `value-terms' to ascribe or disavow value to objects, and they have exemplarily illuminated acts of evaluation in `expert interpretations' of different critical schools (cf. von Heydebrand/Winko 1996).
