
\vspace{.2cm}
\noindent\textbf{\emph{Subproject 2: Normative Issues (K\"oppe \& Klauk/Literary Theory/G\"ottingen/PostDoc}}
\vspace{.1cm}

\noindent This subproject intends to approach (Q2) by examining the influence of \emph{literary appreciation} on what a work of fiction prescribes us to imagine. In particular, we want to contribute to answering the following, more specific question:

\vspace{-.2cm}
\begin{quote}
(Q2') Does appreciation influence what literary works of fiction prescribe us to imagine, and if so, how?
\end{quote}
\vspace{-.2cm}

\noindent The subproject has three interrelated parts. The first part asks for the influence of literary appreciation on the determination of fictional \emph{content}, i.e. on \emph{what} to imagine. This part is mainly analytic in that it tries to uncover the structure and force of arguments in favor of the thesis that appreciation does play a role in the determination of content. The second part is constructive, arguing that the appreciation of certain \emph{narrative-structural} features of literary fictions presupposes an imaginative involvement of a certain \emph{kind}; in this part, we therefore aim at spelling out the influence of literary appreciation on \emph{how} to imagine a particular content. The third part is mainly speculative in focusing on implications of both ventures for the very notion of prescribed imaginings.

In asking for the influence of literary appreciation on what to imagine, the \textbf{first part} of the subproject starts with a very general notion of aesthetic appreciation. According to a widely held view, aesthetic appreciation aims at the appraisal of aesthetic merit, and works of art have aesthetic merit in so far as they offer for intrinsically valuable experiences (Budd 1995; Goldman 2006). This conception of aesthetic appreciation can easily be applied to literary fictions. Take the notorious debate about whether there are ghosts in H. James' \emph{The Turn of the Screw} (Currie 1991). The claim that there are ghosts that are in contact with the children supposedly makes for a (more) scary reading than the claim that there are no ghosts in the story world, but rather a deluded governess. Other things being equal, the fact that the story is scary amounts to a merit-constituting property of a ghost story. Hence, it seems that the claim that `in the fiction\textsubscript{W}, \emph{p}, makes for a more scary reading of work \emph{W}' counts in favor of the claim `in the fiction\textsubscript{W}, \emph{p}'. Generalizing over this particular case, we get something like the following hypothesis: 

\vspace{-.2cm}
\begin{quote}
(H) A content-ascribing interpretation \emph{I\textsubscript{1}} is (other things being equal) better than its alternative \emph{I\textsubscript{2}}, if imagining in accordance with it leads to an intrinsically more valuable experience. (Cf. Goldman 1995, 102)
\end{quote}
\vspace{-.2cm}

\noindent However, both the details and the validity of (H) are far from clear and need to be carefully scrutinized. For instance, the example discussed suggests that whether the dispositional quality of `allowing for a scary reading' counts as a merit-constituting property of a work depends on the genre (cf. Walton 1970). Suppose that we read \emph{The Turn of the Screw} as a psychological novella instead.{~ }Doesn't, in the context of this reading, the claim that in the fiction, there is a deluded governess (\emph{q}), rather than ghosts, constitute the more rewarding reading experience, and hence counts in favor of the claim `in the fiction\textsubscript{W}, \emph{q}? Also, (H) seems to imply that aesthetic experiences can be compared (cf. Budd 1995, 42f., on the incommensurability of aesthetic value). And, given that the \emph{appropriateness} of experiences of aesthetic value depends on the quality of the interpretations the experiences are based on, (H) moreover seems to imply that the appropriateness of experiences of aesthetic value depends on the degree of value ascribed by the interpretation. Both implications of (H) are clearly problematic. Our first subquestion then becomes:

\vspace{-.2cm}
\begin{quote}
(Q2.1)  Does the realization of aesthetic merit help to determine which of two (otherwise equally well supported) content-ascribing interpretations \emph{I\textsubscript{1}} and \emph{I\textsubscript{2}} is to be preferred, and if so, how?
\end{quote}
\vspace{-.2cm}

\noindent (H) is concerned with the rather simple case of two alternative interpretations which are \emph{ex hypothesi} equally well supported in all other relevant respects. However, interpretations can be more or less convincing according to several different criteria (Strube 1992). For instance, content-ascribing interpretations may be judged according to the purely formal criterion of how many (important) textual features they explain (F\o{}llesdal et al. 2008). Now, suppose that we have two (otherwise equally well supported) incompatible interpretations, \emph{I\textsubscript{1}} and \emph{I\textsubscript{2}}, such that \emph{I\textsubscript{1}} ascribes a merit-constituting property as explained above (say, `in the fiction\textsubscript{W}, \emph{p}'), while \emph{I\textsubscript{2}} ascribes another property (say, `in the fiction\textsubscript{W}, \emph{q}'), thereby having greater explanatory scope with respect to other (important) properties of the work than \emph{I\textsubscript{1}}. Which one should be preferred? Our second subquestion therefore is:

\vspace{-.2cm}
\begin{quote}
(Q2.2)  Do merit-constituting properties trump other criteria for content ascriptions, and if so, how?
\end{quote}
\vspace{-.2cm}

\noindent In the next step, we turn to another notion of appreciation, namely literary appreciation more narrowly construed, and its putative influence on content-ascriptions/prescriptions to imagine. Interpreters are not always guided by the goal of enabling readers to undergo experiences which are intrinsically valuable (Shusterman 1978). For instance, they may also aim at `historical interpretation' (cf. Olsen 2004, 142; on different goals of interpretation in literary studies, cf. Kindt/M\"uller 2003, 212; K\"oppe/Winko 2013). Proponents of historical interpretation take seriously the idea that works of literary fiction are historical artifacts and try to uncover the ways the work was intended to be read by its historical audiences (or the ways it was actually read by these audiences). As we shall argue, appreciating a work of fiction as a historical artifact certainly amounts to appreciating the work \emph{as} a work of fiction. Historical interpretation is not meant to merely \emph{use} the work, say, in order to gain historical knowledge about the world (cf. Eco 1999, 35ff.), but rather the work itself and its properties are at the focus of interest. Thus it seems fair to say that historical interpretation amounts to a form of literary appreciation, and is in the business of uncovering \emph{literary merit} (cf. Lamarque 2009, 171). Our third subquestion thus becomes:

\vspace{-.2cm}
\begin{quote}
(Q2.3)  How are we to decide between two (incompatible) interpretations, \emph{I\textsubscript{1}} and \emph{I\textsubscript{2}}, of work W, such  that \emph{I\textsubscript{1}} involves the claim that `in the fiction\textsubscript{W}, \emph{p}', thereby giving rise to literary merit \emph{M\textsubscript{1}}, and  \emph{I\textsubscript{2}} involves the claim that `in the fiction\textsubscript{W}, \emph{q}', thereby giving rise to literary merit \emph{M\textsubscript{2}}?
\end{quote}
\vspace{-.2cm}

\noindent In working on part one of the subproject, we need to keep two methodological points in mind. Firstly, one may be tempted to claim that (Q2.1), (Q2.2) and (Q2.3) need to be answered on a case to case basis. Literary interpretation is often regarded as holistic insofar as one has to trade the total package of pros and cons of \emph{I\textsubscript{1}} off against the total package of pros and cons of \emph{I\textsubscript{2}} \ldots \emph{I\textsubscript{n}} (K\"oppe 2008b, 86f.). The results of Subproject 1 will hopefully contribute to our understanding of the matter. But even if holism gives us the correct picture with respect to content-ascriptions (both \emph{de iure} and \emph{de facto}), we certainly want to know what types of reasons guide, and ought to guide, our decisions in typical cases. Secondly, maybe there is no best interpretation to be had, but rather only optimal ones (in the sense of Currie 2003, 293). We turn to some implications this may have for the notion of prescribed imaginings below, as part three of this subproject.

\textbf{Part two} of the current subproject asks for the influence of literary appreciation on \emph{how} to imagine. It does so by identifying three narrative-structural features the appreciation of which will be argued to make special claims on the imagination. Thus a case will be made that certain narrative structural features not only normatively constrain the content of our imaginings but also \emph{how} to imagine it. We focus on three such features, namely \emph{unreliable narration}, the distinction between \emph{telling vs. showing}, and \emph{internal focalization}.

Unreliable narration comes in many forms (cf. Kindt 2008; K\"oppe/Kindt 2014, ch. 4.4). One form involves the reader's misinformation about the contours of the story world. Readers of Bierce's \emph{An Occurrence at Owl Creek Bridge} are told that the protagonist escapes from a life threatening situation, only to learn later on in the story that the escape was nothing but the hallucinations of the dying man (St\"uhring 2011). This narrative strategy, then, centrally involves a distinction between what is the case in the fiction and what is but \emph{seemingly} the case in the fiction. However, we surely cannot say that only the former is what is important to appreciating the work. For, arguably, appreciating the short story involves that you first faithfully follow the protagonist's hallucinations (without knowing that they are but hallucinations, that is), that you gain hope and be awaked, and probably disappointed, in the end. So, what do we have here in terms of prescriptions to imagine? In previous work, we have proposed to distinguish between prescriptions to imagine and \emph{prima facie} prescriptions to imagine (K\"oppe/Kindt 2011). Readers are, in other words, mislead by the story concerning what it is that they are prescribed to imagine. This solution leaves untouched the idea that prescriptions to imagine establish fictional facts. But it does not seem to capture the idea that, in order to appreciate the story in accordance with its narrative strategy, one actually \emph{needs} to be mislead concerning what is the case in the fiction. The work, as it were, prescribes us to imagine what it does not prescribe us to imagine. This needs explanation. A variant of this account has been proposed by St\"uhring (2011) who claims that passages of unreliable narration give readers a reason to imagine what (as might only be seen from an all things considered perspective) is not the case in the fiction. Again, however, the details are far from clear; in particular, it is not clear how the accounts relate to what has been identified as a major functional effect of the story, i.e. its propensity of elicit a particular emotionally qualified imaginative response (cf. K\"oppe 2012b). Our research question thus becomes:

\vspace{-.2cm}
\begin{quote}
(Q2.4)  How can we accommodate that in certain cases of unreliable narration literary fictions seem  to prescribe emotionally qualified imaginings that, moreover, do not establish fictional facts?
\end{quote}
\vspace{-.2cm}

\noindent Besides unreliable narration, we shall examine two more narrative-structural features of literary fictions and argue that they have a bearing on the \emph{way} the works ask us to imagine what is the case in the fiction. The second feature under consideration is \emph{internal focalization}, i.e. the telling of a narrative from the perspective of one of its characters (Genette 1980, 185ff.; Klauk/K\"oppe/Onea 2012). Take the sentences `Peter looked out of the window. The cars were green.' Presumably, readers are supposed to imagine that the focal character, Peter, sees the green cars, or that the cars looked green to Peter. Accordingly, we have a necessary condition for internal focalization, spelled out in terms of prescribed imaginings. We might, very roughly, represent it like this: `If a passage of text \emph{t} with a content \emph{c} of a fictional narrative is internally focalized through a character \emph{C}, then readers are prescribed by \emph{t} to imagine that `\emph{C} perceives \emph{c}'.' It is a common assumption, however, that the appreciation of internally focalized passages of text not only involves grasping what is the case in the story world. These passages rather invite you to imaginatively perceive the situation \emph{from the character's perspective} (Lindemann 1987, 6; Habermas 2006, 505; Stanzel 2008, 16). The passages, in other words, invite you to put yourself, imaginatively, in the character's shoes. Thus for the Peter-case, you are invited to
imagine seeing the green cars. But again, we need to spell out the details. Recently, the related claim that point of view shots in cinematic fiction prompt readers to imagine seeing things from the character's point of view has been put to criticism (Choi 2005), and a similar case can be made for the effects of internal focalization, it seems. Hence, our research question is:

\vspace{-.2cm}
\begin{quote}
(Q2.5) Do passages of internal focalization prescribe us to imagine the content of what is said from  the point of view of the focal character, and if so, how? 
\end{quote}
\vspace{-.2cm}

\noindent A similar case will be made for the notorious distinction between \emph{telling vs. showing} modes of presentation in a narrative (Klauk/K\"oppe 2014a; Klauk/K\"oppe 2014c; Klauk/K\"oppe 2015). The gist of passages of \emph{showing} seems to be that readers are invited to \emph{vividly} imagine what the text is about. But this is not more than the beginning of an account of the narrative mode of showing. Obviously, any passage of narrative fiction could be said to invite a rich and multifaceted, hence vivid, imaginative engagement. So what is special about \emph{showing}? Hence, our research question is:

\vspace{-.2cm}
\begin{quote}
(Q2.6) Are passages of \emph{showing} special with regard to the way of the imaginings they prescribe?
\end{quote}
\vspace{-.2cm}

\noindent The first two parts of the subproject identify a number of problems and research questions which have implications for the very notion of prescribed imaginings. Spelling out these implications, and arguing for their relevance to a general theory of prescriptions to imagine in the context of fiction constitutes \textbf{part three} of the subproject, which will be conducted in close cooperation with the Subproject 1 on foundational matters.

If the picture of literary interpretation sketched above is correct, then it might turn out that prescriptions to imagine do not issue from works of fiction \emph{tout court} but need to be construed as being somehow sensitive to types of interpretation. (We can think of an interpretation that seeks to promote intrinsically valuable experiences and `historical interpretation' as two types of interpretation here). This gives rise to the following research question:

\vspace{-.2cm}
\begin{quote}
(Q2.7)  Are prescriptions to imagine sensitive to types of interpretation, and if so, how?
\end{quote}
\vspace{-.2cm}

\noindent There are at least three \emph{prima facie} options to answer this question (cf. Lamarque 1996, 64): Firstly, what a work \emph{W} prescribes to imagine could be said to have no unconditional validity, but may be represented as `hypothetical imperatives': `When you engage in an interpretation of type \emph{T} of \emph{W}, then you are prescribed to imagine \emph{\ldots} by \emph{W}'. It is, however, unclear whether all prescriptions to imagine have this hypothetical form or only some. In other words, do we have different kinds of prescriptions to imagine in the appreciation of literary fictions, some with unconditional force and others conditional upon types of interpretation? Secondly, we might opt for the de-emphasis of prescribed imaginings that are not `apt' to a particular type of interpretation \emph{T}, in the context of engaging in a \emph{T}-type interpretation. Walton recommends to de-emphasize fictional facts that do not fit into certain games of make-believe, meaning that there is a prescription to imagine in these cases, but we somehow do not take it seriously (Walton 1990, 182). This strategy does not touch the unconditional force of prescriptions to imagine. However, it rearranges the relation of interpretation and prescriptions to imagine considerably, for interpretations here do not only aim at determining prescriptions to imagine, but also at their evaluation. As a third strategy, we might opt for a `disjunction', claiming that work \emph{W} may have different story worlds \emph{SW\textsubscript{1}, SW\textsubscript{2}, \ldots}, where each \emph{SW} depends on the type of interpretation \emph{T} adopted. In this picture, prescriptions to imagine have unconditional validity but are relativized to types of interpretation. The pros and cons of these alternatives need to be evaluated, and they need to be supplemented by further options. 

In our exposition of part two of the current subproject, we have considered three examples of narrative-structural features that seem to have an influence on the way of imagining that is prescribed by works of fiction. The notion of `way' here refers to a mixed bag of phenomena: \emph{\`a propos} unreliable narration, we have considered the notion of imagining in an emotionally qualified way (such that we, say, are invited to \emph{confidently} imagine that the protagonist escapes) as well as the distinction between \emph{prima facie} and all-things-considered prescriptions, and the idea that imaginings may be supported with different strengths of reasons. \emph{|`A propos} the narrative modes of internal focalization and showing (vs. telling), we have considered the ways of imagining something from a character's perspective, or imagining something in a vivid manner. Our final research question, then, generalizes over the particularities of these cases:

\vspace{-.2cm}
\begin{quote}
(Q2.8) How do prescriptions to imagine extend beyond the content of what is to be imagined to the  \emph{way} of imagining?
\end{quote}
\vspace{-.2cm}
