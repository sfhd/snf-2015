\vspace{.2cm}
\noindent\textbf{\emph{Subproject 3: Foundational Issues (Dorsch/Philosophy/Fribourg/PostDoc}}
\vspace{.1cm}

\noindent The general goal of this subproject is to answer (Q3) by developing an account of the normativity involved in (NR1). A preliminary task is to show why it is misguided to think that the normative relation between fictional stories and imagination could be construed in parallel to that between factual reports and beliefs. A fictional story is not simply a report that is stripped of its commitment to truth; nor is an imaginative thought simply a belief that is insensitive to matters of fact. The link to truth is not just a marginal aspect but a constitutive feature of reports and beliefs. Once that link is removed, we get something fundamentally different. It is therefore not entirely clear what we are supposed to do when asked to consider the normative relation between reports and beliefs \emph{in abstraction from} their connection to truth.

To get a different perspective on the issue, we suggest to leave the standard comparison with reports and beliefs aside (until the end of the third part) and to try to understand the normative relation (NR1) between fiction and imagination in terms of the nature of imagining (first part). After that, we can ask what makes imagining susceptible to normative guidance in general (second part), before addressing the more specific question of how it could be guided by fictional utterances in particular (third part).

The \textbf{first part} of the subproject is concerned with identifying the nature of imaginative thoughts. In general, thoughts are mental episodes in the stream of consciousness that constitute propositional attitudes. As such, they possess a propositional content which determines certain conditions that can be met or not, and which is true if and only if these conditions are actually met. But not all thoughts are imaginative. Our first research question is thus:

\vspace{-.1cm}
\begin{itemize}[leftmargin=2cm]
\item[(Q3.1)] How do imaginative thoughts differ from other thoughts?
\end{itemize}
\vspace{-.1cm}

\noindent In order to answer this question, we need to consider three important distinctions. First of all, the propositional contents of thoughts may be entertained in either of two ways (Velleman 2000a; Stock 2011). On the one hand, if a content is entertained in a \emph{presentational} way (as in the case of guesses or empirical judgments), it seems to the subject as if the conditions determined by the content are already met (somewhere, sometime). On the other hand, if a content is entertained in a \emph{projectional} way (as in the case of wishes or occurrent desires), it seems to the subject as if the conditions determined by the content are still in need to be met. Our hypothesis is that our imaginative responses to fictional texts involve primarily presentational thoughts. For not only is it open to debate whether there are any projectional forms of imagining (Velleman 2000b; Currie \& Ravenscroft 2002b; Kind 2011). But even if, our main imaginative responses to fiction portray the respective fictional world as being a certain way, and not as to be made a certain way (Velleman 2000b; Martin 2002).

Then, only some propositional attitudes are such that their success constitutively depends on the truth of their contents. Cognitive attitudes like beliefs or judgments, for instance, have the function to be true --- they are successful when they fit the world. Conative attitudes like desires or intentions, on the other hand, have the function to be made true --- they are successful when the world is made to fit them. By contrast, imaginative thoughts seem to have neither a cognitive nor a conative direction of fit (Searle 2002). Although they have propositional contents, their success does not constitutively depend on whether their contents are (made) true or not; or so we would like to argue.

Finally, we want to defend the view that imaginative thoughts differ from other propositional attitudes in that we have direct voluntary control over them (McGinn 2004; Dorsch 2009, 2012a, 2016e). We cannot believe or desire something just because we want to do so (Pink 1996; Shah 2003, 2008; Dorsch 2009). But we are able to imagine something simply by deciding to imagine it. Of course, there are certain conceptual and psychological limits to what we can imagine (Gendler 2000; Dorsch 2012: ch. 13.3, 2016c). But, within these limits, we do have voluntarily control over those imaginative thoughts that we (can) actually form. Moreover, this control is direct in the sense that, once we have decided to imagine something, we do not have to do anything else to execute the decision: we can simply imagine what we want to imagine (Dorsch 2009, 2012: ch. 13). Accordingly, our first working hypothesis is:

\vspace{-.1cm}
\begin{itemize}[leftmargin=2cm]
\item[(H3.1)] What is distinctive of imaginative thoughts is that they are presentational, without a direction of fit, and subject to direct
voluntary control.
\end{itemize}
\vspace{-.1cm}

\noindent We would like to conlude the first part by arguing that, assuming that (H3.1) is indeed true, its truth is best explained by the
following hypothesis:

\vspace{-.1cm}
\begin{itemize}[leftmargin=2cm]
\item[(H3.2)] Imaginative thoughts are intentional mental actions.
\end{itemize}
\vspace{-.1cm}

\noindent To start with, only actions seem to allow for direct voluntary control (Pink 1996). In addition, intentional actions arguably do not possess any direction of fit. For they are just the kind of entities that render conative attitudes successful by making the world fit them. Finally, the idea that imaginative thoughts are mental actions is compatible with the observation that we sometimes imagine something `against our will' (e.g. when we cannot banish a certain thought), given that there is room for akratic, obsessive and similarly irrational actions (Dorsch 2012: ch. 13.3). What is special about the class of actions to which imaginative thoughts belong is, however, that they have propositional contents entertained in a presentational way.

The \textbf{second part} of the subproject uses the results from the first in order to answer the question about how it is possible that imaginative thoughts are open to normative guidance:

\vspace{-.1cm}
\begin{itemize}[leftmargin=2cm]
\item[(Q3.2)] How can imaginative thoughts, understood as intentional actions, be normatively guided?
\end{itemize}
\vspace{-.1cm}

\noindent That imaginative thoughts are intentional mental actions suggests that they are sensitive to considerations related to the value that they might serve to realize or promote. For what generally guides us in deciding what to do are considerations that show the action worth doing (Raz 2011, ch. 4; Shah 2008; Hieronymi 2014). The following hypothesis therefore needs to be discussed and evaluated:

\vspace{-.1cm}
\begin{itemize}[leftmargin=2cm]
\item[(H3.3)] The receptivity of imaginative thoughts to normative guidance is due to the value(s) that they realize or promote.
\end{itemize}
\vspace{-.1cm}

\noindent It is commonly assumed that a given action can serve to realize more than one value at once (Raz 2003; Shpall/Wilson 2012; Schroeder 2012)). In this sense, a particular (string of) imaginative thought(s) may have features that render it both entertaining and instructive, say. But it may also be part of some broader activity which is worthwhile to pursue because it furthers a third value. So the first thing that seems relevant to ask is:

\vspace{-.1cm}
\begin{itemize}[leftmargin=2cm]
\item[(Q3.3)] Is there a value that attaches to imagining as such, and which is common to all (and only) imaginative thoughts (as well as other forms of imagining)?
\end{itemize}
\vspace{-.1cm}

\noindent Our working hypothesis regarding this question is that there is indeed a value which any occurrence of imagining realizes or promotes:

\vspace{-.1cm}
\begin{itemize}[leftmargin=2cm]
\item[(H3.4)] Imaginative thoughts (and experiences, etc.) realize the value of (what may be called) \emph{presentational liberty}.
\end{itemize}
\vspace{-.1cm}

\noindent This value is realized whenever one successfully exercises one's capacity to decide which presentational thoughts to entertain, unburdened by external evidential or perceptual restrictions. (H4) is controversial in a lot of respects and calls for many qualifications. In particular, if presentational liberty does indeed constitute a value for imagining, the following issue needs to be clarified:

\vspace{-.1cm}
\begin{itemize}[leftmargin=2cm]
\item[(Q3.4)] Is presentational liberty an intrinsic or merely an instrumental value?
\end{itemize}
\vspace{-.1cm}

\noindent If it is intrinsic, then we should have a \emph{default} reason to imagine something whenever we are in a position to do so -- that is, a reason that holds unless there is more reason to do something else instead. But it is not obvious at all that this is indeed the case. On the other hand, although the claim that the value of presentational liberty is merely instrumental seems less controversial, it remains to be shown which external values the exercise of presentational liberty qua exercise of presentational liberty may help to realize.

We have to appeal to external values in any case in order to answer another question, which concerns the value of specific imaginings rather than that of imagining in general. That our imagination can be subject to normative guidance does not only mean that we can have reason to \emph{imagine} something rather than \emph{doing something else.} It also implies that we can have reason to imagine \emph{this} rather than \emph{that}. So, it also needs to be queried:

\vspace{-.1cm}
\begin{itemize}[leftmargin=2cm]
\item[(Q3.5)] Which value(s) normatively guide(s) our decision to imagine \emph{this} rather than \emph{that}?
\end{itemize}
\vspace{-.1cm}

\noindent Presentational liberty cannot be the answer to this question, given that it attaches to all imaginative thoughts, irrespective of their particular content, and thus cannot adjudicate between imagining \emph{p} and imagining \emph{q}. If values are to guide us not only in whether to imagine anything at all but also in \emph{what} we imagine, they must somehow relate to the contents of our imaginings. The most simple way of construing the connection, we contend, is in terms of the external goals that particular imaginings serve to achieve:

\vspace{-.1cm}
\begin{itemize}[leftmargin=2cm]
\item[(H3.5)] What normatively guides us in deciding \emph{what} to imagine are the external goals that are instrumentally promoted by imagining.
\end{itemize}
\vspace{-.1cm}

\noindent According to this proposal, what we have reason to imagine in a given situation depends on (a) how worthwhile it is to pursue the external goal that this particular imagination serves to attain, and (b) how well the imagination is suited to attain that goal. Given the great variety of external values that imaginative thoughts (like all kinds of action) can help us to attain, and given the variety of alternative ways in which each of these values might be attained, it seems unlikely that a full systematic account of the reasons for particular imaginings can be provided. But the considerations adduced so far at least specify what shape such an account would have to take.

The \textbf{third part} of the subproject then turns to the following, more concrete question:

\vspace{-.1cm}
\begin{itemize}[leftmargin=2cm]
\item[(Q3.6)] Given the preceding general considerations, how can our imagination be normatively guided \emph{specifically by fictional utterances}?
\end{itemize}
\vspace{-.1cm}
 
\noindent The key to understanding how this is possible is the assumption that fictional utterances, too, have a presentational (rather than projectional) make-up. This assumption, in combination with a few further observations about the presentational-cum-practical nature of imaginative thoughts that were made in the first two parts of the subproject, provides the explanation that we are looking for. Since both fictional utterances and imaginative thoughts are presentational in character, there is a perfectly good sense in which what we imagine can be the same as what fictional utterances present. Moreover, since imaginative thoughts are intentional actions, they can be formed for any suitable, deliberately chosen purpose. In particular, they can be formed with the specific purpose of imagining what is presented by some utterance(s). This suggest the following simple account:

\vspace{-.1cm}
\begin{itemize}[leftmargin=2cm]
\item[(H3.6)] We read/listen to fictional utterances (or texts) with the goal to imagine what those utterances (or texts) present, and the imaginative thoughts we form in pursuing that goal can be said to be normatively guided by the utterances (or texts) insofar as and because we take the goal to be worthwhile pursuing.
\end{itemize}
\vspace{-.1cm}

\noindent On such a view, it is not the fictional utterances that constitute or provide us with reasons in favor of forming the corresponding imaginative thoughts, but rather the considerations that show the underlying goal worthwhile pursuing. Still, the guidance that the utterances provide us with can be said to be normative in the (minimal) sense that it is supported by normative considerations -- if we could not see any good in adapting our imaginative thoughts to what the utterances present, the utterances could not be said to serve as a guide to our imagination in the first place.

Moreover, the sense in which it can be correct or incorrect to form certain imaginative thoughts in response to fictional utterances is only indirectly connected to the reasons we see for imagining what we imagine. For what one imagines in response to a fictional utterance is correct (incorrect) in the first instance when and because it conforms (fails to conform) to what the utterance presents. Presentational conformity is the criterion of correctness that is set by the underlying goal, and the rationality of the goal depends on the considerations that we take to render goal worthwhile of pursuit.

So far almost nothing has been said about what these considerations and corresponding values might be. We suppose that the proposal should be compatible with most standard answers that have been given to the question what the function or value of fiction consists in. Thus, imagining what is presented by fictional utterances (stories) might be taken to be entertaining, emotionally moving, or aesthetically appealing (Lamarque \& Olsen 1994, ch. 17), to be instructive or to provide knowledge of some kind (Lamarque \& Olsen 1994, ch. 17; Feagin 1996; Mikkonen 2013), to strengthen our empathetic capacities and our propensity to do good (Nussbaum 1995; Rorty 2001), to fine-tune our mental capacities and skills (Landy 2012), and so on (or any combination thereof).

A lot more needs to be said to motivate and clarify the account, of course. We only mention two of the many issues that have to be addressed by the subproject.

First, the goal to imagine what is presented by some given utterance(s) can be pursued with regard to any utterances with a presentational make-up, not only fictional utterances. For instance, we can adopt that goal when faced with a factual report (indeed, we could treat a factual report as a fictional story). There is no such flexibility when the attitude concerned is belief. We cannot choose the goal with which to form our beliefs, nor can we deliberately determine the constraints under which to form them. In particular, we cannot decide to believe what some utterance presents us with just because it seems worthwhile to do so. Beliefs are responsive to epistemic considerations related to truth, not to practical considerations related to value (see the discussion before (H3.1)). This is why the normative relation between fictional texts and imaginative thoughts cannot be assimilated to that between factual reports and beliefs.

Second, the proposal presupposes that fictional utterances are presentational and have a content that is determined independently of what we (are supposed to) imagine. The general structure of the account would not change if we dropped that assumption; it would only be more complicated. For instance, if we were to assume, instead, that the content of such utterances is fixed by what the speaker intends us to imagine, the goal with which we form our imaginative thoughts in response to those utterances would have to be the goal of imagining what we are intended by the speaker to imagine. What renders the guidance provided by fictional utterances normative, on this version of the account, is not the speaker's intention but the underlying reasons that speak in favor of pursuing the goal of conforming to this intention.


