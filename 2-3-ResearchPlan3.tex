\vspace{.2cm}
\noindent\textbf{\emph{Subproject 3: Foundational Issues (Dorsch \& Bodrozic/Philosophy/Fribourg/PostDoc}}
\vspace{.1cm}

\noindent The general goal of this subproject is to answer (Q3) by developing an account of the normativity involved in (NR1). This account need not be regarded as a rival to the accounts in terms of speech-act commitments and Gricean communicative intentions considered earlier. It may rather be thought of as a supplement to them. Still, what is problematic about those accounts is that, without relevant supplementation, they seem to suggest that the normative relation between fictional stories and imagination can be construed in parallel to that between factual reports and beliefs. So, a first task is to show why this suggestion is misguided. A fictional story is not simply a report that is stripped of its commitment to truth; nor is an imaginative thought simply a belief that is insensitive to matters of fact. The link to truth is not just a marginal aspect but a constitutive feature of reports and beliefs. Once that link is removed, we get something fundamentally different. It is therefore not entirely clear what we are supposed to do when asked to consider the normative relation between reports and beliefs \emph{in abstraction from} their connection to truth.

To get a different perspective on the issue, we suggest to leave the standard comparison with reports and beliefs aside (*at least until later) and to try to understand the normative relation (NR1) between fiction and imagination in terms of the nature of imagining (first part). After that, we can ask what makes imagining susceptible to normative guidance in general (second part), before addressing the more specific question of how it could be guided by fictional utterances in particular (third part).

Our focus is thereby on imaginative thoughts, given that they constitute our primary access to fictional worlds (e.g. when we imagine that Sherlock Holmes lives in Baker Street 221b). But we expect very similar considerations to apply also to imaginative experiences (e.g. visualising the appearance of a fictional character) and more complex imaginative projects (e.g. imaginatively adopting the perspective of a fictional character; see Dorsch 2012: ch. 1).

The \textbf{first part} of the subproject is concerned with identifying the nature of imaginative thoughts. In general, thoughts are mental episodes in the stream of consciousness that constitute propositional attitudes. As such, they possess a propositional content which determines certain conditions that can be met or not, and which is true if and only if these conditions are actually met. The content of the thought `it rains', say, is such that it is true just in case it does rain. But not all thoughts are imaginative. Our first research question is thus:

\vspace{-.2cm}
\begin{quote}
\textbf{(Q3.1) How do imaginative thoughts differ from other thoughts?}
\end{quote}
\vspace{-.2cm}

\noindent In order to answer this question, we need to consider three important distinctions among thoughts. First of all, the propositional contents of thoughts may be entertained in either of two ways (Velleman 2000a; Stock 2011). On the one hand, if a content is entertained in a \emph{presentational} way (as in the case of guesses or empirical judgments), it seems to the subject as if the conditions determined by the content are already met (somewhere, sometime). On the other hand, if a content is entertained in a \emph{projectional} way (as in the case of wishes or occurrent desires), it seems to the subject as if the conditions determined by the content are still in need to be met. Our hypothesis is that our imaginative responses to fictional texts involve primarily presentational thoughts. For not only is it open to debate whether there are any projectional forms of imagining (Velleman 2000b; Currie \& Ravenscroft 2002b; Kind 2011). But even if, our main imaginative responses to fiction portray the respective fictional world as being a certain way, and not as to be made a certain way (Walton 1990: ch. *; Martin 2002).

Then, only some propositional attitudes are such that their success constitutively depends on the truth of their contents. Cognitive attitudes like beliefs or judgments, for instance, have the function to be true --- they are successful when they fit the world. Conative attitudes like desires or intentions, on the other hand, have the function to be made true --- they are successful when the world is made to fit them. By contrast, imaginative thoughts seem to have neither a cognitive nor a conative direction of fit (Searle 2002). Although they have propositional contents, their success does not constitutively depend on whether their contents are (made) true or not; or so we would like to argue.

Finally, we want to defend the view that imaginative thoughts differ from other propositional attitudes in that we have direct voluntary control over them (McGinn 2004; Dorsch 2009, 2012a, 2016e). We cannot believe or desire something just because we want to do so (Pink 1996; Shah 2003, 2008; Dorsch 2009). But we are able to imagine something simply by deciding to imagine it. Of course, there are certain conceptual and psychological limits to what we can imagine (Gendler 2000; Dorsch 2012: ch. 13.3, 2016c). But, within these limits, we do have voluntarily control over those imaginative thoughts that we (can) actually form. Moreover, this control is direct in the sense that, once we have decided to imagine something, we do not have to do anything else to execute the decision: we can simply imagine what we want to imagine (Dorsch 2009, 2012: ch. 13). Accordingly, our first working hypothesis is:

\vspace{-.2cm}
\begin{quote}
(H3.1) What is distinctive of imaginative thoughts is that they are presentational, without a direction of fit, and subject to direct
voluntary control.
\end{quote}
\vspace{-.2cm}

\noindent We would like to conlude the first part by arguing that, assuming that (H3.1) is indeed true, its truth is best explained by the
following hypothesis:

\vspace{-.2cm}
\begin{quote}
(H3.2) Imaginative thoughts are intentional mental actions.
\end{quote}
\vspace{-.2cm}

\noindent To start with, only actions seem to allow for direct voluntary control (Pink 1996). In addition, intentional actions arguably do not possess any direction of fit (Searle *). For they are just the kind of entities that render conative attitudes successful by making the world fit them. For instance, while our action of raising our arm may satisfy our intention to raise our arm, it cannot itself be said to be successful (or unsuccessful) independently of the success of our intention. Finally, the idea that imaginative thoughts are mental actions is compatible with the observation that we sometimes imagine something `against our will' (e.g. when we cannot banish a certain thought), given that there is room for akratic, obsessive and similarly irrational actions (Dorsch 2012: ch. 13.3). What is special about the class of actions to which imaginative thoughts belong is, however, that they have propositional contents entertained in a presentational way.

The \textbf{second part} of the subproject uses the results from the first in order to answer the question about how it is possible that imaginative thoughts are open to normative guidance:

\vspace{-.2cm}
\begin{quote}
(Q3.2) How can imaginative thoughts, understood as intentional actions, be normatively guided?
\end{quote}
\vspace{-.2cm}

\noindent That imaginative thoughts are intentional mental actions means that their normativity is linked to that of conative attitudes rather than to that of cognitive attitudes -- despite the fact that imaginative thoughts are also presentational. What is sensitive to normative considerations in an intentional action is not the action per se but the underlying intention (or the desire that motivates it). Still, the normative considerations that intentions are sensitive to are essentially related to a feature that pertains to the intended actions, namely the value that the actions might serve to realize or promote. In other words, what generally guides us in forming our intentions and decisions are considerations that show the intended actions worth doing (Raz 2011, ch. 4; Shah 2008; Hieronymi 2014). So, given that imaginative thoughts constitute a particular form of intentional action, the following hypothesis needs to be discussed and evaluated:

\vspace{-.2cm}
\begin{quote}
(H3.3) The receptivity of imaginative thoughts to normative guidance is due to the value(s) that they realize or promote.
\end{quote}
\vspace{-.2cm}

\noindent It is commonly assumed that a given action can serve to realize more than one value at once (Raz 2003). This is so not only because an action can have several features that constitute mutually independent goods, but also because both actions and values can stand in various systematic dependency relations to other actions and values (for instance, one action can be part of another action, just as one value can be a species of some other, more generic value, etc. -- cf. Raz 2003; Shpall \& Wilson 2012; Schroeder 2012). Thus, a particular (string of) imaginative thought(s) may have features that render it both entertaining and instructive, say. But it may also be part of some broader activity which is worthwhile to pursue because it furthers a third value. Hence, there might be many values that are candidates for being normative guides for our imagination, given that they are promoted by imagining something. Given this context, it is necessary to ask:


\vspace{-.2cm}
\begin{quote}
(Q3.3) Is there a value that attaches to imagining as such, and which is common to all (and only) imaginative thoughts (as well as other forms of imagining)?
\end{quote}
\vspace{-.2cm}


\noindent Our working hypothesis regarding this question is that there is indeed a value which any *successful act of imagining realizes or promotes:

\vspace{-.2cm}
\begin{quote}
(H3.4) *Successful imaginative thoughts realize the value of (what may be called) \emph{presentational liberty}.
\end{quote}
\vspace{-.2cm}

\noindent This value is realized whenever one successfully exercises one's capacity to *autonomously/for oneself decide which presentational thoughts to entertain, unburdened by external evidential or perceptual restrictions. (H4) is controversial in a lot of respects and calls for many qualifications. In particular, provided that presentational liberty does indeed constitute a value for imagining, the following issue needs to be clarified:

\vspace{-.2cm}
\begin{quote}
(Q3.4) Is presentational liberty an intrinsic or merely an instrumental value?
\end{quote}
\vspace{-.2cm}

\noindent If it is intrinsic, then we should have a \emph{default} reason to imagine something whenever we are in a position to do so -- that is, a reason that holds unless there is more reason to do something else instead. But it is not obvious at all that this is indeed the case. On the other hand, although the claim that the value of presentational liberty is merely instrumental seems less controversial, it remains to be shown what other kinds of value the exercise of presentational liberty qua exercise of presentational liberty may help to realize.


We have to appeal to *extrinsic values, in any case, in order to answer another question, which concerns the value of specific imaginings rather than that of imagining in general. That our imagination can be subject to normative guidance does not only mean that we can have reason to \emph{imagine} something rather than \emph{doing something else.} It also implies that we can have reason to imagine \emph{this} rather than \emph{that}. So, it also needs to be queried:

\vspace{-.2cm}
\begin{quote}
(Q3.5) Which value(s) normatively guide(s) our decision to imagine \emph{this} rather than \emph{that}?
\end{quote}
\vspace{-.2cm}


\noindent Presentational liberty cannot be the answer to this question as well, given that it attaches to any imaginative thought, irrespective of its particular content, and thus cannot adjudicate between imagining \emph{p} and imagining \emph{q}. If values are to guide us not only in whether to imagine anything at all but also in \emph{what} we imagine, they must somehow relate to the contents of our imaginings. The most simple way of construing the connection, we contend, is in terms of the extrinsic goals that particular imaginings serve to achieve:

\vspace{-.2cm}
\begin{quote}
(H3.5) What normatively guides us in deciding \emph{what} to imagine are the extrinsic goals that are instrumentally promoted by imagining.
\end{quote}
\vspace{-.2cm}

\noindent According to this proposal, what we may have reason to imagine in a given situation depends on (a) how worthwhile it is to pursue the goal that this particular imagination serves to attain, and (b) how well the imagination is suited to attain that goal. Given the great variety of values that imaginative thoughts (like all kinds of action) can help us to attain, and given the variety of alternative ways in which each such value might be attained, it seems unlikely that a full systematic account of the reasons for particular imaginings can be provided. But the considerations adduced so far at least specify what shape such an account would have to take.




