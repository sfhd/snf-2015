
\vspace{.2cm}
{\textbf{2.4.  }}\textbf{Cooperation, Time Schedule and Benchmarks}
\vspace{.2cm}

\noindent There will be one extended meeting per academic year between the groups from Fribourg and G\"ottingen, as well as a yearly conference with international guests and fortnightly local meetings in Fribourg. Also, there will be further regular meetings between parts of the Fribourg and G\"ottingen groups based on ongoing collaborations of Kindt and K\"oppe who are both members of the `Arbeitsstelle f\"ur Theorie der Literatur' (G\"ottingen University) and of the network `Foundational Concepts of Narratology', as well as co-editors of the \emph{Journal of Literary Theory}. Finally, we will make extensive use of Skype, develop an online discussion forum and use similar means for weekly research exchanges.

\vspace{.5cm}
\setlength{\parindent}{-20pt}
\resizebox{17cm}{!}{\tiny
\begin{tabular}{|c|c|c|c|}
\hline
&&&\\
& \textbf{Subproject 1} & \textbf{Subproject 2} & \textbf{Subproject 3} \\
&&&\\
\hline
&&&\\
\textbf{First Year (2016--17)} & Part 1: preparation of the `analytic manual' \& Part 2: the paraphrase of selections from the corpus & Part 1: the influence of appreciation on what to imagine & Part 1: imaginative thoughts as intentional actions\\
&&&\\
\hline
&&&\\
\textbf{Second Year (2017--18)} & Part 2: empirical work on the catalogue of questions & Part 2: the influence of appreciation on how to imagine & Part 2: the normativity of imagining in general\\
&&&\\
\hline
&&&\\
\textbf{Third Year (2018--19)} & Writing up the PhD thesis & Part 3: the implications for prescriptions to imagine & Part 3: the normativity of imagining in response to fiction\\
&&&\\
\hline
&&&\\
\textbf{Planned Output} & PhD thesis \& one journal article & 6 journal articles & 6 journal articles\\
&&&\\
\hline
\end{tabular}}
\setlength{\parindent}{10pt}

\vspace{.2cm}
\noindent \textbf{2.5.  The Significance of the Research}
\vspace{.2cm}

\noindent Analyzing the norms that govern our imaginative involvement as well as the aesthetic appreciation that is based on this involvement promises to have an impact on both theoretical and practical aspects of literary studies. As to the theoretical side, it makes an important contribution to our understanding of the norms of literary interpretation (which is still at the heart of literary studies as conducted in academia), and hence contributes to the methodology of the discipline. Moreover, by focusing on the ways in which certain narrative structures can be said to guide our imaginings, we get a better understanding of these very structures --- especially since many of them must be taken to be response dependent properties of texts, thereby centrally involving the imagination. As to the practical side of literary studies, both Subproject 1 and Subproject 2 will enhance our understanding of certain prevalent interpretive disputes. (As a side effect, we also hope to gain a better understanding of `Der Sandmann' and other fictions under consideration.)
 
Moreover, by centrally involving the notion of appreciation, all three subprojects bridge the gap between literary studies as an academic discipline and reading novels as conducted by the `layman'. While literary studies is often concerned with some more or less rigorous, `scientific' notion of analysis, it should turn to furthering our understanding of what makes literary fictions valuable to us -- or so we believe. 

The main contribution of Subproject 3 to the philosophical literature on fiction and imagination is that it highlights and illuminates the normativity of an important relation that is often taken for granted but not well understood. It is often claimed that fictional texts invite, prescribe, direct, guide, demand, or ask for an imaginative response. Yet it is rarely discussed where the normative authority of fictional texts over our imaginings could derive from. The project is supposed to fill this theoretical gap. It thus opens a new field of discussion that is supposed to supplement the existing debates on the relation between fiction and imagination. 

In particular, the observation that imaginative thoughts are intentional mental actions with a presentational make-up adds four new elements to the existing debates: (i) that the source of the authority of fictional texts lies in the values that we see in engaging with them; (ii) that imaginative thoughts can be normatively guided, despite lacking a direction of fit; (iii) that imagining is inseparably linked to presentational liberty; and (iv) the normative relation between factual reports and belief cannot serve as a foil for understanding the normative relation between fictional texts and imagination. These elements put the standard accounts of (NR1) into a new perspective. 



