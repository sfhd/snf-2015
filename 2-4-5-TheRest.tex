
\vspace{.2cm}
{\textbf{2.4.  }}\textbf{Time Schedule and Main Benchmarks}
\vspace{.2cm}

\noindent There will be at least four extended workshops per year wheren the groups from Fribourg and G\"ottingen will meet, as well as fortnightly local meetings in Fribourg. In addition, we will make extensive use of Skype, online discussion forums and similar means for weekly research exchanges.

\vspace{.5cm}
\setlength{\parindent}{-20pt}
\resizebox{17cm}{!}{\tiny
\begin{tabular}{|c|c|c|c|}
\hline
&&&\\
& \textbf{Subproject 1} & \textbf{Subproject 2} & \textbf{Subproject 3} \\
&&&\\
\hline
&&&\\
\textbf{First Year (2016--17)} & Preparation of the `analytic manual' and the paraphrase of selections from the corpus & Part 1: the influence of appreciation on what to imagine & Part 1: imaginative thoughts as intentional actions\\
&&&\\
\hline
&&&\\
\textbf{Second Year (2017--18)} & Empirical work on the catalogue of questions & Part 2: the influence of appreciation on how to imagine & Part 2: the normativity of imagining in general\\
&&&\\
\hline
&&&\\
\textbf{Third Year (2018--19)} & Writing up the PhD thesis & Part 3: the implications for prescriptions to imagine & Part 3: the normativity of imagining in response to fiction\\
&&&\\
\hline
&&&\\
\textbf{Planned Output} & PhD thesis \& one journal article & 6 journal articles & 6 journal articles\\
&&&\\
\hline
\end{tabular}}

\textbf{2.5.  The Significance of the Research}

The main contribution of the frst subproject to the already existing literature on value- and attitude-based accounts of reasons is that it develops and defends an important observation that is usually just mentioned in passing, without much further elucidation or justifcation (e.g. Kolodny 2007). The claim in question is that reasons and principles of rationality pertain by default to different perspectives that we may adopt – reasons to the frst-personal perspective that we take up in deliberation about what to do (or to believe), and principles of rationality to the third-personal perspective that we adopt when assessing the deliberative performance of others (or, for that matter, ourselves). The thorough investigation of this difference between the two aspects of normativity concerned is philosophically signifcant not the least because it helps us to recognise that philosophers, who take attitudes governed by principles of rationality to be the source of reasons, make the error of not properly distinguishing between the two perspectives, as well as to explain why they tend to make this error.

The second subproject is of importance because it investigates the signifcance of a certain epistemic constraint on which reasons for action we have, which is not often discussed, at least not with respect to practical reasons (notable exceptions are Raz 2011 and Kolodny 2013). The constraint in question is that reasons are subjective in so far as something can be a practical reason for us only if we have some form of access to it. Otherwise, it could not fgure in our frst-personal deliberations. The discussion of this constraint is necessary since it seems to be in tension with the further assumption that reasons, due their origin in values, are objective, thus questioning the very distinction between value-based (or objective) and attitude-based (or subjective) reasons. One part of our resulting contribution to the debate is to work out in detail how best to dispel the worry about the objectivity of value-based, but also essentially accessible reasons; and another, to investigate how exactly the border separating value- and attitude-based approaches to reasons should be redrawn in the light of our results concerning the objectivity of value-based reasons.

Finally, the most interesting aspect of the third subproject is that it argues for a kind of reasons – namely commitment-based reasons – that are, strictly speaking, not value-based, but none the less similar enough to value-based reasons in order to be easily accommodated by value-based accounts of reason, without falling victim to the objections typically raised against hybrid views that assume both value- and attitude-based reasons. The idea is that commitment- involving attitudes may serve as independent sources of reasons is not entirely new, but has not been very often discussed or defended (notable exceptions are Chang 2013, Shpal 2013 and 2014, and Betzler 2014c). Moreover, to our knowledge, it has not yet been argued that commitment-based reasons resemble value-based reasons in that they also depend on some value, albeit on a past instance of value rather than a current instance.
