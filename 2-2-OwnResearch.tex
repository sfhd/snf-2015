
\vspace{.2cm}
\noindent\textbf{\large 2.2. State of Own Research \& Institutional Setting}
\vspace{.2cm}

\noindent \textbf{\emph{Subproject 1.}} Tom Kindt's past and present research is closely linked to the proposed research, especially with respect to the historical appraisal of literary studies' development since the 19\textsuperscript{th} century and a systematic clarification of interpretation as one of the discipline's fundamental activities. Concerning the first of these two issues, several of Kindt's publications examine instructive episodes of the history of the humanities and of literary criticism in particular, in order to illuminate the structural architecture of approaches to literature and their linkage to general cultural developments (cf. e.g. Kindt/M\"uller 2005, 2008). The second topic mentioned is reflected in the theoretical focal point of Kindt's publications. On the one hand, in a series of studies, he has elucidated central aspects and traditions of scholarly interpretation. The publications in question explicate their object as a normative practice composed of different lower level operations (like description, classification, explanation, evaluation, etc.) and develop proposals for distinguishing between theories, methods and heuristics of interpretation (Kindt/K\"oppe 2008; Kindt/M\"uller 2003; Kindt 2007; K\"oppe/Kindt 2014; Kindt 2015a). A recent contribution of this type relates these distinctions to the interpretive debate about E.T.A. Hoffmanns `Der Sandmann' and outlines a text-based explanation of why the novella has induced such a long-standing succession of controversial statements (Kindt 2016). On the other hand, Kindt's emphasis on meta-theoretical issues becomes manifest in a couple of publications that analyze the interpretive usage of highly disputed critical concepts like `implied author' (Kindt/M\"uller 2006, 2011), `unreliable narration' (Kindt 2008; K\"oppe/Kindt 2014), `fantasy' (Kindt 2011), or `literary epoch' (Kindt 2015b). 

The Institute for German and Literary Studies at the University of Fribourg provides an excellent research surrounding for the subproject. Due to the proven collaboration between its literary studies- and linguistics-division and to the research foci of its members over the last years, the institute is known for stimulating theoretical and meta-critical contributions to literary criticism and for advanced attempts to combine hermeneutic and empirical approaches to text and discourse analysis.


\vspace{.2cm}
\noindent \textbf{\emph{Subproject 2.}} Tilmann K\"oppe's work in literary theory has focused on several topics at the intersection of philosophical aesthetics and literary studies. In particular, he has worked on fiction as a source of knowledge (K\"oppe 2007; 2008b; 2009b; 2011), the methodological foundations of literary interpretation (Dennerlein/K\"oppe/Werner 2008; K\"oppe/Winko 2011; K\"oppe 2012a), and problems in the theory of fiction (K\"oppe 2005; Gertken/K\"oppe 2009; K\"oppe 2009a; 2014a; 2014b; 2014c; 2014d; Klauk/K\"oppe 2014b; Klauk/K\"oppe/Rami 2014). In his current research, K\"oppe focuses on the textual foundations of `higher-order' narrative features such as \emph{narrative closure}, \emph{narrative unreliability} (K\"oppe/Kindt 2011) or the concept of a \emph{narrator} (K\"oppe/St\"uhring 2011; 2015), and \emph{narrative distance}/\emph{telling} vs. \emph{showing} (Klauk/K\"oppe 2014a; 2014c; 2015). 

In addition to K\"oppe's expertise, G\"ottingen is an ideal location for this subproject due to the longstanding close cooperation between philosophy and literary studies established at the Courant Research Centre `Textstrukturen' where Tilmann K\"oppe is head of the research group `Analytic Literary Theory'. Also, the German Department hosts the `Arbeitsstelle f\"ur Theorie der Literatur' (ATL), of which Tom Kindt and Tilmann K\"oppe are members. The ATL features a strong focus on analytic literary theory (cf. K\"oppe 2008a; K\"oppe/ Winko 2010) which makes it a natural cooperation partner for the planned project.

\vspace{.2cm}
\noindent \textbf{\emph{Subproject 3.}} Fabian Dorsch has worked and published on the three major topics of the proposed research: imagination, aesthetic appreciation, and normativity. He is one of the few contemporary philosophers to have published a book on the philosophy of the imagination (Dorsch 2012a) and is preparing another book for Routledge (Dorsch 2016e). While the first defends an account of the nature and unity of the various forms of imagining that takes imaginings to be intentional mental actions, the second deals with the difference between imaginative experiences and imaginative thoughts and their distinctive roles in our acquisition of knowledge and our appreciation of art and fiction. In addition, his work on the imagination include a co-edited volume on memory and imagination (Dorsch \& Macpherson 2016) and articles on visualising (Dorsch 2010), imaginative and emotional responses to fiction (Dorsch 2011), Hume's views on the imagination (Dorsch 2013b, 2016c), extended imaginative projects (Dorsch 2016a), and imagination-based accounts of pictorial experience (Dorsch 2015b, 2016b). Dorsch's research on aesthetic appreciation is focused on the nature of aesthetic experience (Dorsch 2000), the justification of aesthetic judgements (Dorsch 2007, 2013a), and the relevance of empirical investigations for aesthetic appreciation and normativity (Dorsch 2012b, 2014); while his work on normativity deals with reasons for belief and action (Dorsch 2009), perceptual justification (Dorsch 2010, 2016d) and visualization-based knowledge (Dorsch 2015a). Subproject 3 is a natural continuation of Dorsch's previous work on imagination, appreciation and normativity.

Fribourg is a perfect location for this subproject, given that the philosophy department there hosts not only the EXRE Centre for Research on Mind and Normativity, but also Dorsch's current research group `The Normative Mind', which investigates the normativity of beliefs, intentions, actions and evaluations. The group is also a founding member of the European Normativity Network (other members include Barcelona, Paris, Southampton and Oslo). 
