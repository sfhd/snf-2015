
\vspace{.2cm}
\noindent\textbf{\large 2.2. State of Own Research}
\vspace{.2cm}

Subproject 1 is closely linked to some of the core areas of Tom Kindt's past and present research, especially with respect to the historical appraisal of literary studies' development since the 19\textsuperscript{th} century and to a systematic clarification of interpretation as one of the discipline's fundamental activities. Concerning the first of these two issues, several of Kindt's publications examine instructive episodes of the history of the humanities and of literary criticism in particular, in order to illuminate the structural architecture of approaches to literature and their linkage to general cultural developments (cf. e.g. Kindt/M\"uller 2004, 2005, 2008). The second topic mentioned is reflected in the theoretical focal point of Kindt's publications. On the one hand, in a series of studies, he has elucidated central aspects and traditions of scholarly interpretation. The publications in question explicate their object as a normative practice composed of different lower level operations (like description, classification, explanation, evaluation, etc.) and develop proposals for distinguishing between theories, methods and heuristics of interpretation (Kindt/K\"oppe 2008; Kindt/M\"uller 2003; Kindt 2007; K\"oppe/Kindt 2014; Kindt 2015a). A recent contribution of this type relates these distinctions to the interpretive debate about E.T.A. Hoffmanns `Der Sandmann' and outlines a text-based explanation of why the novella has induced such a long-standing succession of controversial statements (Kindt, in press). On the other hand, Kindt's emphasis on meta-theoretical issues becomes manifest in a couple of publications that analyze and attempt to optimize the interpretive usage of highly disputed critical concepts like `implied author' (Kindt/M\"uller 2006, 2011), `unreliable narration' (Kindt 2008; K\"oppe/Kindt 2014), `fantasy' (Kindt 2011), or `literary epoch' (Kindt 2015b).


\vspace{.2cm}
\noindent Tilmann K\"oppe's work in literary theory has focused on several topics at the intersection of philosophical aesthetics and literary studies. In particular, he has worked on fiction as a source of knowledge (K\"oppe 2007; 2008; 2009; 2011), the methodological foundations of literary interpretation (Dennerlein/K\"oppe/Werner 2008; K\"oppe/Winko 2011; K\"oppe 2012a), and Problems in the theory of fiction K\"oppe 2005; Gertken/K\"oppe 2009; K\"oppe 2009a; 2014; 2014a; 2014b; 2014c; Klauk/K\"oppe 2014b; Klauk/K\"oppe/Rami 2014). In his current research, K\"oppe focuses on the textual foundations of `higher-order' narrative features such as \emph{narrative closure}, \emph{narrative unreliability} (K\"oppe/Kindt 2011) or the concept of a \emph{narrator} (K\"oppe/St\"uhring 2011; 2015), and \emph{narrative distance}/\emph{telling} vs. \emph{showing} (Klauk/K\"oppe 2014; 2014a; 2015). Importantly for the present research proposal, this research builds on the assumption that our understanding of the relation between higher-order narrative phenomena such as a `narrator' can be enhanced by considering the theory of fiction.

\vspace{.2cm}
\noindent Fabian Dorsch has worked and published on the three major topics of the proposed research: imagination, aesthetic appreciation, and normativity. He is one of the few contemporary philosophers to have published a book on the (analytic) philosophy of the imagination (Dorsch 2012a) and has another book on the topic due for publication with Routledge (Dorsch 2016e). While the first book addresses the issue of how best to account for the nature and unity of the various forms of imagining and defends the view that imaginings are intentional mental actions, the second deals with the difference between imaginative experiences and imaginative thoughts and their distinctive roles in our acquisition of knowledge and our appreciation of art (including literature, paintings and other fictional works). In addition, his work on the imagination include a co-edited volume on sensory memory and imagination (Dorsch \& Macpherson 2016), as well as articles on the nature of visualising (Dorsch 2010), our imaginative and emotional responses to fictional works (Dorsch 2011), Hume's views on the imagination (Dorsch 2013b, 2016c), extended imaginative projects such as daydreaming or mind-wandering (Dorsch 2016a), and imagination-based accounts of pictorial experience (Dorsch 2005, 2016b). Dorsch's research on (the normativity of) aesthetic appreciation is focused on the nature of aesthetic experience (Dorsch 2000), the justification of aesthetic judgements and evaluations (Dorsch 2013a), and the relevance of empirical investigations for appreciation and aesthetic normativity (Dorsch 2012b, 2014); while his work on normativity deals with the responsiveness of occurrent beliefs and mental actions to reasons (Dorsch 2009), rational motivation (Dorsch 2016d), the justificatory force of perceptual experiences (Dorsch 2013c, Dorsch \& Dutant 2016), the possibility of visualization-based knowledge (Dorsch 2015) and, as already mentioned, the justification of aesthetic responses.

* Bodrozic / Klauk