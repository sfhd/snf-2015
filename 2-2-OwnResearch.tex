

\noindent\textbf{\large 2.2. State of Own Research}

\vspace{.2cm}
\noindent Tilmann K\"oppe's work in literary theory has focused on several topics at the intersection of philosophical aesthetics and literary studies. In particular, he has worked on fiction as a source of knowledge (K\"oppe 2007; 2008; 2009; 2011), the methodological foundations of literary interpretation (Dennerlein/K\"oppe/Werner 2008; K\"oppe/Winko 2011; K\"oppe 2012a), and Problems in the theory of fiction K\"oppe 2005; Gertken/K\"oppe 2009; K\"oppe 2009a; 2014; 2014a; 2014b; 2014c; Klauk/K\"oppe 2014b; Klauk/K\"oppe/Rami 2014). In his current research, K\"oppe focuses on the textual foundations of `higher-order' narrative features such as \emph{narrative closure}, \emph{narrative unreliability} (K\"oppe/Kindt 2011) or the concept of a \emph{narrator} (K\"oppe/St\"uhring 2011; 2015), and \emph{narrative distance}/\emph{telling} vs. \emph{showing} (Klauk/K\"oppe 2014; 2014a; 2015). Importantly for the present research proposal, this research builds on the assumption that our understanding of the relation between higher-order narrative phenomena such as a `narrator' can be enhanced by considering the theory of fiction.

The subproject in closely linked to some of the core areas of Tom Kindt's past and present research, especially, to the endeavors of a historical appraisal of literary studies' development since the 19\textsuperscript{th} century and a systematic clarification of interpretation as one of the discipline's fundamental activities. Pursuing the first of these two enterprises, several of Kindt's publications examine instructive episodes of the history of the humanities and of literary criticism in particular, in order to illuminate the structural architecture of approaches to literature and their linkage to general cultural developments (cf., e.g., Kindt/M\"uller 2004, 2005, 2008). The second endeavor mentioned is reflected in the theoretical focal point of Kindt's publications: In a series of studies, he has, on the one hand, aimed to elucidate central aspects and traditions of scholarly interpretation. The publications in question explicate their object as a normative practice composed of different lower level operations (like description, classification, explanation, evaluation, etc.) and develop proposals for distinguishing between theories, methods and heuristics of interpretation (Kindt/K\"oppe 2008; Kindt/M\"uller 2003; Kindt 2007; K\"oppe/Kindt 2014; Kindt 2015a). A recent contribution of this type relates these distinctions to the interpretive debate about E.T.A. Hoffmanns ``Der Sandmann'' and outlines a text-based explanation of why the novella has induced such a long-standing succession of controversial statements (Kindt, in press). On the other hand, Kindt's emphasis on meta-theoretical issues becomes manifest in a couple of publications that analyze and attempt to optimize the
interpretive usage of highly disputed critical concepts like `implied author' (Kindt/M\"uller 2006, 2011), `unreliable narration' (Kindt 2008; K\"oppe/Kindt 2014), `fantasy' (Kindt 2011), or `literary epoch' (Kindt 2015b).

* Dorsch / Bodrozic / Klauk