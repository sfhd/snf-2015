
\vspace{.4cm}
\noindent\textbf{Subproject 2: Normative Issues (K\"oppe/Literary Theory/G\"ottingen/PostDoc)}
\vspace{.2cm}

\noindent\textbf{\emph{Theories of Truth in Fiction}}\emph{.} Current theories of fiction maintain that what is the case in the story world of a literary work of fiction is centrally determined by what recipients are prescribed to imagine by the work (Evans 1982; Walton 1990; Everett 2013; Walton 2013). Fictional facts are, in short, facts to be imagined. An indispensable part of most, if not all, interpretations as conducted in literary studies consists in determining what is the case in the story world of particular works of fiction. The philosophical debate on the determination of fictional content has identified several general difficulties for this interpretive endeavor. In particular, it is clear that the fact that a work contains `\emph{p}' is not necessary for \emph{p} being the case in the story world. In the fiction, Sherlock Holmes has two lungs, although the novels do not say so explicitly. Readers draw all kinds of inferences from what is said in a story, and they assume that certain (but certainly not all) facts from the real world hold in the story world, whether the text says so or not (Currie 1990, 60; Bareis 2009). Yet there is a broad consensus that there are norms constraining content ascriptions. In particular, it seems that there are certain `import rules' that govern what must or must not be imported from the real world to a given fictional world. However, there are different candidates for such import rules, and with respect to particular cases, they lead to more or less intuitively plausible, and sometimes conflicting, results (cf. K\"oppe 2014d for a recent summary). This has led some to maintain that there is no general `principle of generation' for fictional facts (Walton 1990, ch. 4; Lamarque/Olsen 1994, 94; Zipfel 2001, 90), while others maintain that one may keep searching for a meta principle that allows us to choose among the existing principles (Petterson 1993, 91), or that such a principle would at least explain the sometimes considerable agreement amongst interpreters (cf. Livingston 2005, 192). Still others maintain that the search for overarching principles is misguided from the start, and that the principles invoked in the philosophical discussion are not what guides the interpretations that are actually being conducted by literary scholars (Lamarque 1990; 1996, ch. 4).

\vspace{.2cm}
\noindent\textbf{\emph{Theory of Fiction and Narratology}}. Philosophical and narratological research so far is just beginning to acknowledge the importance of the theory of fiction for both our appreciation of particular narrative features of works of fiction and our understanding of the accordant narratological concepts (cf. Mart\'inez/Scheffel 2003; Walsh 2007; Bareis 2008; K\"oppe/Kindt 2014). This is surprising given that many narratological concepts are defined in terms of what is the case in a fiction, and given that the notion of truth in fiction is elucidated in the theory of fiction in turn. An example for this that is gaining more and more attention in the literature is the notion of a fictional narrator (Kania 2005; Diehl 2009; K\"oppe/St\"uhring 2011; 2015). But, arguably, a similar case can be made for narratological notions such as \emph{unreliable narration} (K\"oppe/Kindt 2011). Moreover, notions such as the distinction between \emph{telling} and \emph{showing} as modes of narrative presentation (cf. Klauk/K\"oppe 2015) seem to designate response-dependent phenomena such that their elucidation involves the evocation of particular attitudes on the part of the reader. Thus, the \emph{showing} mode of presentation is often said to consist in the propensity of a text to give readers the impression to be imaginarily `present on the scene' (cf. Mart\'inez/Scheffel 1999, 47f.). Since large branches of the theory of fiction involve reference to the `fictive stance', i.e. a particular reception mode on the part of the reader (cf. Wolterstorff 1980; Walton 1990; Lamarque/Olsen 1994), it seems natural to inquire into the relationships between these aspects of the theory of fiction and of narratology.


